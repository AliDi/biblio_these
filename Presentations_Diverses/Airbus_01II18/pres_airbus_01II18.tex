\documentclass[10pt,xcolor=x11names,compress, notes=show]{beamer}% pour l'impression, tout n'apparait qu'une fois \documentclass[handout,12pt]{beamer}

%\documentclass[xcolor=x11names]{beamer}
%\usepackage[scaled]{helvet}
\usepackage[round]{natbib}
\usepackage[utf8x]{inputenc}
%\usepackage{ucs}
\usepackage[french]{babel}
\usepackage{todonotes}
\usepackage{tikz}
\usepackage{color}
%\usepackage{subfigure}
%\usepackage[]{geometry}
\usepackage{changepage}
\usetikzlibrary{calc}

\usepackage{bm}
\usepackage{pifont} %pour les symbole sympa \ding{nb}
\usepackage[export]{adjustbox}
\usepackage{subcaption}

\usepackage{pdfpages}
\setbeamertemplate{navigation symbols}{} 
%\usepackage{palatino}

%pour le theme
%\usetheme{CambridgeUS}

%\usetheme{Goettingen}
%\useinnertheme{default}
%\useoutertheme[subsection=false]{miniframes} %%pour avoir le défilement en en-tête des diapos par section
\setbeamertemplate{blocks}[rounded][shadow=true]
\setbeamercolor{block title}{fg=DeepSkyBlue4,bg=DeepSkyBlue4!10}
\setbeamercolor{block title alerted}{bg=DeepSkyBlue4!0} 
\setbeamercolor{block title example}{bg=DeepSkyBlue4!20}
%\setbeamercolor*{lower separation line head}{bg=DeepSkyBlue4} 

\setbeamerfont{title like}{shape=\scshape}
\setbeamercolor{frametitle}{fg=DeepSkyBlue4}
\setbeamercolor{title}{fg=DeepSkyBlue4}
\setbeamercolor{itemize item}{fg=black}
\setbeamercolor{itemize subitem}{fg=black}
\setbeamercolor{toc}{fg=DeepSkyBlue4}
\usepackage{amsmath,mathtools}
\usefonttheme[onlymath]{serif}
\usepackage{bm}

\setbeamertemplate{frametitle}{\vspace{0.5cm}\hspace{-0.9cm} \insertframetitle}
\setbeamerfont{frametitle}{size=\Large}

\setlength{\fboxrule}{1pt}

%couleur table des matières
\usepackage{hyperref}
\hypersetup{colorlinks=true, linkcolor=DeepSkyBlue4}

\setbeamertemplate{caption}{\raggedright\insertcaption\par}

%Mettre la section courante en titre de diapo (pour champ de titre non-vide)
%\addtobeamertemplate{frametitle}{\frametitle{\insertsubsectionhead}}{}

%\addtobeamertemplate{footline}{\hspace{11cm} \insertframenumber}
\setbeamertemplate{footline}[frame number]
%\newcommand{\tr}[1]{\prescript{t\hspace{-0.08cm}}{}{#1}}

\usepackage{wrapfig}

%pour les plots matlab en tikz
\usepackage{pgfplots} 
\pgfplotsset{compat=newest}

\usepackage{multicol}
\setlength{\columnsep}{0cm}

\usepackage{pifont} %pour les symbole sympa \ding{nb}


%%% Page de titre
%======================
\author{J. {Antoni}, A. {Dinsenmeyer} and Q. {Leclère} \\ Laboratoire Vibrations Acoustique}
\title{Denoising of the CSM}
\subtitle{}
\date{February 2018}
%\titlegraphic{}

\begin{document}
%%%		Title
%======================
\begin{frame}
	\maketitle
\end{frame}

%%% Context
%======================
\section{Context}
\begin{frame}{\insertsectionhead}

Unwanted random noise : 
\begin{itemize}
	\item electronic, ambient, flow-induced,...
	\item short correlation lengths
\end{itemize}
~\\
Existing denoising methods : 
\begin{itemize}
	\item Physical removal : mic recession, porous treatment, ...
	\item Use of a background noise measurement
	\item Wavenumber filtering
\end{itemize}
\end{frame}

\begin{frame}{CSM properties}
	$$\bm{S}_{p} = \frac{1}{N_s} \sum_i  \bm{p}_i\bm{p}_i'$$
	\begin{itemize}
	   	 \item Hermitian (conjugate symmetric)
	   	 \item Positive semidefinite (nonnegative eigenvalues)
	\end{itemize}~\\
	
	$$\underbracket[0.5pt]{\bm{S}_p}_{\text{measured CSM}} =\underbracket[0.5pt]{ \bm{S}_a}_{\text{signal of interest}} +\underbracket[0.5pt]{\bm{S}_n}_{\text{unwanted noise}}$$
	
	\begin{itemize}
	     \item Signal CSM : one eigenvalue for one incoherent source
	    \item Noise CSM : off-diagonal elements $\rightarrow 0$ with averaging
	\end{itemize}
\end{frame}

%%% Algorithms
%==========================
\section{Denoising algorithms}
\begin{frame}{\insertsectionhead}
	\begin{block}{Diagonal reconstruction}
		\begin{equation*}
		    \text{maximize~} \sum_i \sigma_{n_i} \text{~~subject to~~} \bm{S}_{pp}-\operatorname{diag}(\bm{\sigma}_n)\geq 0
		\end{equation*}
		Solved with CVX Matlab toolbox.
	\end{block}	
	\begin{block}{Robust Principal Component Analysis}
		\begin{equation*}
			\text{minimize~} \|\bm{{S}}_{a} \|_* + \lambda \| \bm{{S}}_{n} \|_1  \text{~~~~subject to~~~~}  \bm{{S}}_{a} +  \bm{{S}}_{n} = \bm{S}_{p}
		\end{equation*}
		Solved with proximal gradient algorithm.
	\end{block}	
	\begin{block}{Probabilistic Factorial Analysis}
		\begin{equation*}
		        \bm{p} = \bm{L}\operatorname{diag}(\bm{\alpha})\bm{C} +\operatorname{diag}(\bm{\sigma})\bm{\epsilon}
		\end{equation*}
		\begin{equation*}
		   \bm{L},\bm{C},\bm{\sigma}\sim \mathcal{N}(0,\bm{\Omega}_{L,C,\sigma}^2)~~~~\text{and}~~~~\bm{\alpha}\sim \mathcal{N}(\bm{\mu}_{\alpha},\bm{\Omega}_{\alpha}^2)
		\end{equation*}
		Solved with Gibbs sampling algorithm.
	\end{block}
\end{frame}

%%% Results
%==============================
\section{Results}
\begin{frame}{Test case}
	\begin{multicols}{2}
		\begin{itemize}
			\item frequency : 15 kHz
			\item 20 monopoles
			\item 93 receivers
			\item SNR : 10 dB
			\item $10^4$ snapshots
		\end{itemize}	
	\end{multicols}	\vspace{-16pt}
	\begin{itemize}
    		\item heterogeneous noise : SNR 10 dB lower on 10 random receivers
	\end{itemize}
	\vfill
	\centering
	% This file was created by matlab2tikz.
%
%The latest updates can be retrieved from
%  http://www.mathworks.com/matlabcentral/fileexchange/22022-matlab2tikz-matlab2tikz
%where you can also make suggestions and rate matlab2tikz.
%
\begin{tikzpicture}

\begin{axis}[%
width=4.3cm,
height=3cm,
at={(0cm,0cm)},
scale only axis,
xmin=-1,
xmax=1,
tick align=outside,
xlabel style={at={(0.15,0)}, font=\color{white!15!black},font=\scriptsize},
xlabel={x (m)},
ymin=-1,
ymax=0,
ylabel style={at={(0.7,-0.05)}, font=\color{white!15!black},font=\scriptsize},
ylabel={y (m)},
zmin=-1,
zmax=1,
zlabel style={at={(-0.08,0.5)}, font=\color{white!15!black},font=\scriptsize},
zlabel={z (m)},
view={-125.1}{10},
axis background/.style={fill=white},
axis x line*=bottom,
axis y line*=left,
axis z line*=left,
ytick={0,-0.5,-1},
xmajorgrids,
ymajorgrids,
zmajorgrids,
legend style={at={(0.2,1)}, anchor=south west, legend cell align=left, align=left, draw=white!15!black},
ticklabel style={font=\scriptsize}
]
\addplot3 [color=white, line width=1.0pt, draw=none, mark size=1.2pt, mark=o, mark options={line width=0.8pt,solid, colorAlice}]
 table[row sep=crcr] {%
0	-1	0\\
0.106660701334476	-1	0\\
0.0533000007271767	-1	0.092399999499321\\
-0.0533000007271767	-1	0.092399999499321\\
-0.106660701334476	-1	-9.31999988296184e-09\\
-0.0533000007271767	-1	-0.092399999499321\\
0.0533000007271767	-1	-0.092399999499321\\
0.196912005543709	-1	0\\
0.174356892704964	-1	0.091499999165535\\
0.111858800053596	-1	0.162055402994156\\
0.023700000718236	-1	0.195476293563843\\
-0.0697999969124794	-1	0.184115901589394\\
-0.147390797734261	-1	0.130576804280281\\
-0.191190093755722	-1	0.0471000000834465\\
-0.191190093755722	-1	-0.0471000000834465\\
-0.147390693426132	-1	-0.130576804280281\\
-0.0697999969124794	-1	-0.184115901589394\\
0.023700000718236	-1	-0.195476293563843\\
0.111858800053596	-1	-0.162055402994156\\
0.174356997013092	-1	-0.091499999165535\\
0.315122187137604	-1	0\\
0.287878394126892	-1	0.128171801567078\\
0.210857897996902	-1	0.23418140411377\\
0.09740000218153	-1	0.299699008464813\\
-0.0329000018537045	-1	0.313395887613297\\
-0.15756119787693	-1	0.272903800010681\\
-0.254939287900925	-1	0.18522410094738\\
-0.308236002922058	-1	0.0654999986290932\\
-0.308236002922058	-1	-0.0654999986290932\\
-0.25493910908699	-1	-0.185224294662476\\
-0.157560899853706	-1	-0.272903889417648\\
-0.0329000018537045	-1	-0.313395887613297\\
0.09740000218153	-1	-0.299698889255524\\
0.210858106613159	-1	-0.234181299805641\\
0.287878513336182	-1	-0.128171503543854\\
0.463685810565948	-1	0\\
0.432374089956284	-1	0.167502596974373\\
0.342667907476425	-1	0.312383085489273\\
0.206682503223419	-1	0.415074497461319\\
0.0428000018000603	-1	0.461707711219788\\
-0.126893594861031	-1	0.445984899997711\\
-0.279433101415634	-1	0.370029211044312\\
-0.394233614206314	-1	0.244099095463753\\
-0.455790609121323	-1	0.0851999968290329\\
-0.455790609121323	-1	-0.0851999968290329\\
-0.394233614206314	-1	-0.244099095463753\\
-0.279433101415634	-1	-0.370029211044312\\
-0.126893594861031	-1	-0.445984899997711\\
0.0428000018000603	-1	-0.461707800626755\\
0.206682503223419	-1	-0.415074497461319\\
0.342667788267136	-1	-0.312383085489273\\
0.432374089956284	-1	-0.167502596974373\\
0.672533094882965	-1	0\\
0.636093378067017	-1	0.218371197581291\\
0.530723094940186	-1	0.413078397512436\\
0.367840707302094	-1	0.563022196292877\\
0.165096998214722	-1	0.651953816413879\\
-0.0555000007152557	-1	0.670236110687256\\
-0.270153611898422	-1	0.615887820720673\\
-0.455494403839111	-1	0.494798600673676\\
-0.591475307941437	-1	0.32009020447731\\
-0.663360714912415	-1	0.110695198178291\\
-0.663360595703125	-1	-0.110695503652096\\
-0.591475129127502	-1	-0.320090502500534\\
-0.455494105815887	-1	-0.494798898696899\\
-0.270153313875198	-1	-0.615887880325317\\
-0.0555000007152557	-1	-0.670236110687256\\
0.165097206830978	-1	-0.651953816413879\\
0.367841005325317	-1	-0.563022017478943\\
0.530723392963409	-1	-0.413078010082245\\
0.636093497276306	-1	-0.218370899558067\\
0.899999976158142	-1	0\\
0.863543689250946	-1	0.253559291362762\\
0.757128119468689	-1	0.486576706171036\\
0.589374601840973	-1	0.680174589157104\\
0.3738734126091	-1	0.818668782711029\\
0.128083303570747	-1	0.890839278697968\\
-0.128083497285843	-1	0.890839278697968\\
-0.373873591423035	-1	0.818668723106384\\
-0.589374780654907	-1	0.68017452955246\\
-0.757128179073334	-1	0.486576706171036\\
-0.863543689250946	-1	0.253559112548828\\
-0.899999976158142	-1	-2.92999999373933e-07\\
-0.863543629646301	-1	-0.253559499979019\\
-0.757128119468689	-1	-0.486576914787292\\
-0.589374482631683	-1	-0.680174827575684\\
-0.373873203992844	-1	-0.818668901920319\\
-0.128083094954491	-1	-0.890839278697968\\
0.128083497285843	-1	-0.890839278697968\\
0.373873591423035	-1	-0.818668723106384\\
0.589375019073486	-1	-0.68017441034317\\
0.757128417491913	-1	-0.486576408147812\\
0.863543689250946	-1	-0.253559112548828\\
};
% \addlegendentry{Receivers}

\addplot3 [color=white, line width=1.0pt, draw=none, mark size=1.2pt, mark=diamond*, mark options={solid, fill=red, red}]
 table[row sep=crcr] {%
0.179001071858587	0	-1.19981625468941\\
0.181211485347157	0	-1.07351980682736\\
0.183421898835727	0	-0.94722335896532\\
0.185632312324296	0	-0.820926911103277\\
0.187842725812866	0	-0.694630463241235\\
0.190053139301436	0	-0.568334015379192\\
0.192263552790006	0	-0.442037567517149\\
0.194473966278576	0	-0.315741119655107\\
0.196684379767145	0	-0.189444671793064\\
0.198894793255715	0	-0.0631482239310213\\
0.201105206744285	0	0.0631482239310213\\
0.203315620232855	0	0.189444671793064\\
0.205526033721424	0	0.315741119655107\\
0.207736447209994	0	0.442037567517149\\
0.209946860698564	0	0.568334015379192\\
0.212157274187134	0	0.694630463241235\\
0.214367687675704	0	0.820926911103277\\
0.216578101164273	0	0.94722335896532\\
0.218788514652843	0	1.07351980682736\\
0.220998928141413	0	1.19981625468941\\
};
% \addlegendentry{Monopoles}

\end{axis}
\end{tikzpicture}%
\end{frame}

\begin{frame}{\insertsectionhead}	
	\resizebox{1.1\textwidth}{!}{
	\begin{minipage}{1.2\textwidth}
		\centering		
		\hspace{-1cm}% This file was created by matlab2tikz.
%
%The latest updates can be retrieved from
%  http://www.mathworks.com/matlabcentral/fileexchange/22022-matlab2tikz-matlab2tikz
%where you can also make suggestions and rate matlab2tikz.
%
\definecolor{mycolor1}{rgb}{0.00000,0.44700,0.74100}%
\definecolor{mycolor2}{rgb}{0.85000,0.32500,0.09800}%
\definecolor{mycolor3}{rgb}{0.92900,0.69400,0.12500}%
\definecolor{mycolor4}{rgb}{0.49400,0.18400,0.55600}%
%
\begin{tikzpicture}

\begin{axis}[%
width=5cm,
height=5cm,
at={(0.4in,0.577in)},
scale only axis,
xmin=1,
xmax=93,
xlabel style={font=\color{white!15!black}},
xlabel={Rank of $\bm{S}_{aa}$},
ymin=-25,
ymax=0,
ylabel style={font=\color{white!15!black}},
ylabel={Relative error $\delta$ (dB)},
axis background/.style={fill=white},
legend style={legend cell align=left, align=left, draw=white!15!black},
ticklabel style={font=\footnotesize}
]
\addplot [color=mycolor1, line width=1.0pt]
  table[row sep=crcr]{%
1	-13.1442964301748\\
2	-13.6639625284353\\
3	-14.2625228046774\\
4	-14.9133257850728\\
5	-14.6141885062269\\
6	-14.5233493184842\\
7	-14.7096173725865\\
8	-15.071897260059\\
9	-15.3436135835982\\
10	-15.0320872657786\\
11	-14.8763635557484\\
12	-14.9101566662145\\
13	-14.9761351258711\\
14	-15.0348795551697\\
15	-14.9989675579731\\
16	-15.5430130979618\\
17	-15.4075293695441\\
18	-15.3662136438344\\
19	-15.3536346755599\\
20	-15.2578768324126\\
21	-15.6171099889027\\
22	-15.5299878107985\\
23	-15.207837935637\\
24	-15.5708482377215\\
25	-15.7241791137753\\
26	-15.3272682596613\\
27	-15.762915049281\\
28	-15.5961753213362\\
29	-15.7103964161745\\
30	-15.6712338635308\\
31	-15.9292084751775\\
32	-15.676752321649\\
33	-16.0487889252335\\
34	-16.1808040831508\\
35	-15.7951556260462\\
36	-15.8682661830455\\
37	-16.0069308457919\\
38	-16.1225640435412\\
39	-16.3597999141064\\
40	-16.2951437302316\\
41	-15.637794338108\\
42	-16.3784375212482\\
43	-16.1360769539573\\
44	-16.2612094739016\\
45	-16.9709582689305\\
46	-16.2486060558589\\
47	-16.1821376767254\\
48	-16.6639975375873\\
49	-15.6939831071071\\
50	-17.0581779157398\\
51	-16.2998940323438\\
52	-16.7931054822789\\
53	-16.581356326445\\
54	-17.4737719369992\\
55	-16.4922764314446\\
56	-16.9416264678226\\
57	-16.6160747707713\\
58	-16.7876052941449\\
59	-16.2841719899733\\
60	-16.8764134812354\\
61	-18.0363496099649\\
62	-17.1866986025047\\
63	-17.2997632218667\\
64	-16.6996614160921\\
65	-16.7538374816412\\
66	-17.6226488165345\\
67	-17.078369788401\\
68	-17.334906407646\\
69	-17.3538439059417\\
70	-17.0357650178993\\
71	-16.7904276825689\\
72	-16.9768770341977\\
73	-16.5800047936518\\
74	-15.8037778348193\\
75	-16.2673045001822\\
76	-17.5756530548359\\
77	-15.4960595849571\\
78	-15.4565347925625\\
79	-14.9800397576197\\
80	-14.6599536833219\\
81	-14.293496118402\\
82	-14.2013217082315\\
83	-14.4323095357939\\
84	-13.7243684816235\\
85	-13.404856121827\\
86	-13.448288552778\\
87	-13.4468715063505\\
88	-12.2840560397979\\
89	-12.416329637629\\
90	-11.8540938391377\\
91	-12.5704255843017\\
92	-13.1042189453916\\
93	-12.5266380618653\\
};
%\addlegendentry{Hald}

\addplot [dashed,color=mycolor2, line width=1.0pt]
  table[row sep=crcr]{%
1	-20.1925699742343\\
2	-20.7737121861252\\
3	-21.5515251871168\\
4	-21.418148936107\\
5	-22.4777352373212\\
6	-20.4456448254355\\
7	-22.1536042233647\\
8	-22.760462224135\\
9	-22.6543441236452\\
10	-22.7028156095382\\
11	-22.3206505342227\\
12	-22.3027819960848\\
13	-22.4974574036717\\
14	-22.6243116701211\\
15	-21.7042512031255\\
16	-22.7527870628653\\
17	-22.3610764191817\\
18	-21.9634969739474\\
19	-22.6702519641807\\
20	-22.422975086315\\
21	-21.3235416665985\\
22	-22.7251846216884\\
23	-21.8976447976481\\
24	-22.3689702451976\\
25	-22.4256696500955\\
26	-23.7852607176454\\
27	-21.8300960235752\\
28	-22.0713098832706\\
29	-21.2970176735505\\
30	-22.1031542727271\\
31	-22.1880175914744\\
32	-22.6568532726401\\
33	-22.7780198229113\\
34	-22.3568317008856\\
35	-21.4127230171606\\
36	-21.7899802275602\\
37	-21.8902424946874\\
38	-21.9496960815571\\
39	-22.5341605071503\\
40	-21.9931383530169\\
41	-21.4302317608449\\
42	-21.4316027791809\\
43	-21.564222135896\\
44	-22.0601580601922\\
45	-21.1430833954135\\
46	-22.2474035178478\\
47	-20.8562484350795\\
48	-20.3667944955386\\
49	-21.2007085879699\\
50	-20.8993319735656\\
51	-20.946692654761\\
52	-20.509997723215\\
53	-20.1229614185055\\
54	-20.1219093963103\\
55	-20.042635840299\\
56	-20.8848966374834\\
57	-20.5817576075529\\
58	-19.8423839177247\\
59	-19.5950129395062\\
60	-19.6772678650691\\
61	-19.2833291544963\\
62	-18.7494544348551\\
63	-19.0058425430321\\
64	-18.3235003842975\\
65	-17.8802610168952\\
66	-18.1814065389414\\
67	-17.52847646565\\
68	-16.853030618596\\
69	-17.2906602920078\\
70	-17.3209247654828\\
71	-16.5310846301333\\
72	-16.4494477650051\\
73	-16.5160772341006\\
74	-16.1721650793417\\
75	-14.9774786603036\\
76	-15.8272535867573\\
77	-15.376708366389\\
78	-14.4157521990305\\
79	-14.1853992517373\\
80	-14.0734589574178\\
81	-13.9917890853604\\
82	-14.3755197700198\\
83	-13.3474904586229\\
84	-13.1891015940138\\
85	-12.9712198407164\\
86	-12.741141422751\\
87	-12.7007572870846\\
88	-12.4532438585937\\
89	-12.0348552873319\\
90	-11.7188835103106\\
91	-12.5257634747949\\
92	-12.3325849595395\\
93	-12.1495221178681\\
};
%\addlegendentry{RPCA, $\lambda_{opt}$}

\addplot [color=mycolor2, line width=1.0pt]
  table[row sep=crcr]{%
1	-20.1910994966877\\
2	-20.5417064730134\\
3	-21.4063373662689\\
4	-21.403144678293\\
5	-22.3959252098554\\
6	-20.110598230931\\
7	-22.066623626455\\
8	-22.6949496744732\\
9	-22.3707086916575\\
10	-22.6575055049746\\
11	-22.0711277212712\\
12	-21.860050091972\\
13	-22.4723184282873\\
14	-22.1193874174511\\
15	-21.62088422302\\
16	-22.2501115832348\\
17	-21.6598629630356\\
18	-21.1420315216433\\
19	-21.3031451388734\\
20	-20.4703758333499\\
21	-20.2618701655679\\
22	-20.8819177428219\\
23	-19.189454698891\\
24	-19.88330395389\\
25	-19.422389775278\\
26	-18.8189554799803\\
27	-17.6087768579812\\
28	-17.5297694651481\\
29	-16.8758193455656\\
30	-16.5111323795986\\
31	-15.6699011233491\\
32	-16.206647459279\\
33	-13.9865141900377\\
34	-9.23794643183106\\
35	-8.60322905862865\\
36	-7.80901544291779\\
37	-6.97036595409524\\
38	-5.63999633170459\\
39	-5.35962849806442\\
40	-4.91831877287392\\
41	-4.3312819220529\\
42	-4.12447953649406\\
43	-3.72434645656639\\
44	-3.32277573290846\\
45	-3.16342028889348\\
46	-3.21239866384865\\
47	-3.04269514398894\\
48	-2.75373038403519\\
49	-2.69733632044462\\
50	-2.47710990778074\\
51	-2.31002418019347\\
52	-2.25202638222817\\
53	-1.94073074510028\\
54	-1.94098487589105\\
55	-1.79913801878156\\
56	-1.76375365845953\\
57	-1.87750011785278\\
58	-1.70376499709454\\
59	-1.64039589134954\\
60	-1.6272776605568\\
61	-1.47149767417307\\
62	-1.48707286094084\\
63	-1.35132018311896\\
64	-1.31484672271347\\
65	-1.30480986730715\\
66	-1.1978647377598\\
67	-1.15973938033834\\
68	-1.05739052839756\\
69	-1.06122961034806\\
70	-0.975725646149897\\
71	-1.00116742879117\\
72	-1.00551900603197\\
73	-1.00955999052691\\
74	-0.999147034168788\\
75	-0.97440231784608\\
76	-0.934717218221778\\
77	-0.933109521160196\\
78	-0.834439154049714\\
79	-0.891612997844411\\
80	-0.859291402888223\\
81	-0.823091700417474\\
82	-0.819540161357988\\
83	-0.757481702411767\\
84	-0.8348788962137\\
85	-0.793433860136455\\
86	-0.760672439995686\\
87	-0.716744802193044\\
88	-0.689498988299274\\
89	-0.666184433357129\\
90	-0.711337436518335\\
91	-0.649934582607169\\
92	-0.692478514050601\\
93	-0.671923386219329\\
};
%\addlegendentry{RPCA, $\lambda=1/\sqrt{M}$}

\addplot [color=mycolor4, line width=1.0pt]
  table[row sep=crcr]{%
1	-20.2900001240088\\
2	-21.2798239129561\\
3	-21.694937059233\\
4	-22.615396401514\\
5	-21.6542127822801\\
6	-21.7698768015351\\
7	-22.2909043106533\\
8	-22.3971993966937\\
9	-21.5144594674575\\
10	-22.1244574339974\\
11	-21.820503334719\\
12	-22.8126000819393\\
13	-22.276785704162\\
14	-22.5293319785549\\
15	-22.8575612533576\\
16	-21.8261511659491\\
17	-22.5054686880501\\
18	-23.4847245132626\\
19	-23.1630889418601\\
20	-23.0668293383553\\
21	-21.3473867185223\\
22	-22.715305114681\\
23	-22.1005248459073\\
24	-23.1714339531157\\
25	-22.2899952425701\\
26	-21.8181789201591\\
27	-21.1922671634997\\
28	-22.8590710387284\\
29	-22.7138228119459\\
30	-21.8530387990841\\
31	-21.7608813723093\\
32	-22.2932686443456\\
33	-22.2744663408225\\
34	-22.5168603966006\\
35	-22.3907788250785\\
36	-22.1235837963021\\
37	-22.2012616869109\\
38	-22.019984349095\\
39	-21.6110127013098\\
40	-22.1892467689572\\
41	-22.7310989268951\\
42	-21.5829778394305\\
43	-22.2287408380513\\
44	-21.3008608935329\\
45	-21.6803216466148\\
46	-22.4091713617005\\
47	-22.4856319968549\\
48	-20.8974446502267\\
49	-21.9017946189661\\
50	-21.835304259894\\
51	-21.9752707717224\\
52	-21.6298312702111\\
53	-21.3687009370543\\
54	-21.0272899598468\\
55	-20.855673549995\\
56	-21.8195232197792\\
57	-20.5248266501447\\
58	-20.3098930323301\\
59	-20.8454312753133\\
60	-19.4567312288398\\
61	-18.9421727629078\\
62	-18.8542469548518\\
63	-18.9945501792648\\
64	-18.0389533432993\\
65	-18.4121066284945\\
66	-17.7919344026314\\
67	-17.3367819822451\\
68	-17.1903925608035\\
69	-16.6162990666\\
70	-16.1858184227672\\
71	-16.0465253857036\\
72	-15.9231500226554\\
73	-15.9099692220881\\
74	-14.9550678963556\\
75	-15.0969153767693\\
76	-14.3777326210676\\
77	-14.718574274869\\
78	-14.4511021128736\\
79	-13.4791013884182\\
80	-13.2227480340031\\
81	-12.8845896066157\\
82	-13.0780365852667\\
83	-12.81435737105\\
84	-12.672217098703\\
85	-12.0054113933546\\
86	-11.9042423098645\\
87	-11.9031999848198\\
88	-11.8617901440646\\
89	-11.4767629388028\\
90	-10.8141938304937\\
91	-11.1667240336738\\
92	-11.2706008752939\\
93	-10.9240187564493\\
};
%\addlegendentry{MCMC}

\end{axis}
\end{tikzpicture}%
		\hspace{-0.2cm}% This file was created by matlab2tikz.
%
%The latest updates can be retrieved from
%  http://www.mathworks.com/matlabcentral/fileexchange/22022-matlab2tikz-matlab2tikz
%where you can also make suggestions and rate matlab2tikz.
%
\definecolor{mycolor1}{rgb}{0.00000,0.44700,0.74100}%
\definecolor{mycolor2}{rgb}{0.85000,0.32500,0.09800}%
\definecolor{mycolor3}{rgb}{0.92900,0.69400,0.12500}%
\definecolor{mycolor4}{rgb}{0.49400,0.18400,0.55600}%
%
\begin{tikzpicture}

\begin{axis}[%
width=3.5cm,
height=3.5cm,
at={(0.4in,0.577in)},
scale only axis,
xmin=-10,
xmax=10,
xlabel style={font=\color{white!15!black}},
xlabel={SNR (dB)},
ymin=-25,
ymax=5,
ylabel style={font=\color{white!15!black}},
%ylabel={Relative error on diag($\bm{S}_{aa}$) (dB)},
axis background/.style={fill=white},
legend style={legend cell align=left, align=left, draw=white!15!black},
ticklabel style={font=\scriptsize},
inner sep=1ex
]
\addplot [color=mycolor1, line width=1.0pt]
  table[row sep=crcr]{%
-10	4.17011664013284\\
-9	3.80036669875062\\
-8	2.21867525289206\\
-7	1.56274066910244\\
-6	0.338977174124358\\
-5	-0.469367952227489\\
-4	-1.52209663395538\\
-3	-2.5457421602982\\
-2	-3.31792674691234\\
-1	-4.66520380937804\\
0	-5.16944316646276\\
1	-6.45199550963716\\
2	-7.63483910688664\\
3	-8.64724391881657\\
4	-9.89797867554265\\
5	-10.884686596658\\
6	-11.2051199554889\\
7	-12.1045355574493\\
8	-13.5628138632752\\
9	-14.5870802662865\\
10	-15.1150716903682\\
};
%\addlegendentry{Hald}

\addplot [dashed,color=mycolor2, line width=1.0pt]
  table[row sep=crcr]{%
-10	-8.78697469523395\\
-9	-8.92341793946406\\
-8	-10.059991472494\\
-7	-11.1740116325522\\
-6	-11.6810643076936\\
-5	-12.3970792793589\\
-4	-14.4375847757424\\
-3	-14.6595402945219\\
-2	-15.7521917889396\\
-1	-16.0097832520357\\
0	-16.5351813633546\\
1	-16.930614593703\\
2	-18.5595743983638\\
3	-19.2207021497336\\
4	-19.3332401156144\\
5	-20.227138285006\\
6	-21.1123316551254\\
7	-21.5697521726293\\
8	-21.9798554050257\\
9	-22.7222428651795\\
10	-23.0924479416379\\
};
%\addlegendentry{RPCA, $\lambda_{opt}$}

\addplot [color=mycolor2, line width=1.0pt]
  table[row sep=crcr]{%
-10	-6.64625402401924\\
-9	-6.8539269426754\\
-8	-7.36220498381084\\
-7	-8.09652003898477\\
-6	-8.84542337380954\\
-5	-9.43112186068687\\
-4	-10.1720135077941\\
-3	-10.6057996032459\\
-2	-11.3823604814007\\
-1	-11.8960032516792\\
0	-12.41335485798\\
1	-14.1275449133311\\
2	-14.7870913536967\\
3	-14.4340197546362\\
4	-16.0996580892364\\
5	-16.7714944375056\\
6	-17.9255772171091\\
7	-17.6761819826806\\
8	-18.763357064702\\
9	-20.7180676097236\\
10	-20.6360471775278\\
};
%\addlegendentry{RPCA, $\lambda=1/\sqrt{M}$}

\addplot [color=mycolor4, line width=1.0pt]
  table[row sep=crcr]{%
-10	-10.6609672164242\\
-9	-11.4186311963878\\
-8	-12.1302716668479\\
-7	-12.7172378968614\\
-6	-13.226444407063\\
-5	-13.919373991279\\
-4	-14.6095349159151\\
-3	-15.3045819949814\\
-2	-15.9798215369256\\
-1	-16.6537686270195\\
0	-17.3311280987057\\
1	-17.4277566684269\\
2	-17.7514000917573\\
3	-17.9861691726516\\
4	-18.4086999568209\\
5	-19.1756127508303\\
6	-20.1553514278741\\
7	-21.0630781483235\\
8	-21.991263780742\\
9	-22.5909954587676\\
10	-23.1813751555255\\
};
%\addlegendentry{MCMC}

\end{axis}
\end{tikzpicture}%
		\hspace{-0.5cm}% This file was created by matlab2tikz.
%
%The latest updates can be retrieved from
%  http://www.mathworks.com/matlabcentral/fileexchange/22022-matlab2tikz-matlab2tikz
%where you can also make suggestions and rate matlab2tikz.
%
\definecolor{mycolor1}{rgb}{0.00000,0.44700,0.74100}%
\definecolor{mycolor2}{rgb}{0.85000,0.32500,0.09800}%
\definecolor{mycolor3}{rgb}{0.92900,0.69400,0.12500}%
\definecolor{mycolor4}{rgb}{0.49400,0.18400,0.55600}%
%
\begin{tikzpicture}

\begin{axis}[%
width=3.5cm,
height=3.5cm,
at={(0.4in,0.629in)},
scale only axis,
xmode=log,
xmin=10,
xmax=50000,
xminorticks=true,
xlabel style={font=\color{white!15!black}},
xlabel={Number of snapshots $N_s$},
ymin=-28,
ymax=-3,
%ylabel style={font=\color{white!15!black}},
%ylabel={Relative error on diag($\bm{S}_{aa}$) (dB)},
axis background/.style={fill=white},
legend style={legend cell align=left, align=left, draw=white!15!black},
ticklabel style={font=\scriptsize},
inner sep=.7ex
]
\addplot [color=mycolor1, line width=1.0pt]
  table[row sep=crcr]{%
10	-4.59441645273614\\
12	-4.6858586422924\\
15	-5.65859712322859\\
19	-4.47327208245575\\
24	-4.80983662126747\\
30	-5.55648471951207\\
37	-5.29933765691523\\
46	-5.56473725009553\\
57	-5.95956993488616\\
71	-5.68778776573777\\
89	-6.00556714883711\\
110	-6.72611636107327\\
137	-6.4998159900808\\
171	-7.11643693320473\\
213	-7.33275492776967\\
265	-7.88575698634865\\
329	-8.55666864360413\\
410	-8.74980497292512\\
510	-9.14074661859439\\
634	-9.97831018497\\
789	-10.3016670851439\\
981	-10.2137763132833\\
1221	-10.5100977151492\\
1519	-11.594725718724\\
1889	-12.0861632627917\\
2350	-12.4360200517069\\
2924	-13.1604505400906\\
3638	-13.4666526366958\\
4526	-13.5132729280192\\
5630	-14.1853026790791\\
7004	-14.851994320544\\
8714	-15.6524302735701\\
10841	-15.6015675440842\\
13486	-16.5082071802014\\
16778	-16.51895765298\\
20873	-16.7482574633298\\
25968	-17.5527591122373\\
32306	-17.9449629923704\\
40191	-18.8457266107001\\
50000	-19.0495703389226\\
};
%\addlegendentry{Hald}

\addplot [dashed,color=mycolor2, line width=1.0pt]
  table[row sep=crcr]{%
10	-8.02820395711242\\
12	-7.11140540571627\\
15	-7.86339680049474\\
19	-8.40121381336962\\
24	-9.95587494631373\\
30	-9.87945806289129\\
37	-10.8818586667373\\
46	-10.6712022912463\\
57	-11.3779723416618\\
71	-11.8255956650116\\
89	-11.7150441789938\\
110	-12.1123663159343\\
137	-13.0543605168726\\
171	-13.1048376800142\\
213	-13.9252158407737\\
265	-15.5475923394248\\
329	-14.8495139800057\\
410	-15.0797395689319\\
510	-16.1188290240635\\
634	-16.7298697950491\\
789	-17.2993440783736\\
981	-17.726720423701\\
1221	-18.9607950082555\\
1519	-18.606981164552\\
1889	-18.9573097969854\\
2350	-19.0424017922529\\
2924	-20.4607981637207\\
3638	-20.2039859866077\\
4526	-21.3848151197662\\
5630	-21.3813537733568\\
7004	-21.0257942895012\\
8714	-21.6544161082101\\
10841	-23.5679185242935\\
13486	-23.0184406093682\\
16778	-23.3149881841635\\
20873	-24.2391803291538\\
25968	-24.5040314170313\\
32306	-24.7215487587778\\
40191	-24.4548185818861\\
50000	-25.0457024767968\\
};
%\addlegendentry{RPCA, $\lambda_{opt}$}

\addplot [color=mycolor2, line width=1.0pt]
  table[row sep=crcr]{%
10	-6.34613807707525\\
12	-4.0660933241545\\
15	-4.67696844346617\\
19	-5.55857093069267\\
24	-5.83696664753308\\
30	-7.34374521867811\\
37	-7.90920058221677\\
46	-9.23467964717356\\
57	-9.91680958847782\\
71	-11.0290109421562\\
89	-11.5056567892518\\
110	-11.9263458875946\\
137	-12.7673878687945\\
171	-12.8584845042257\\
213	-13.9021361924712\\
265	-15.3718522356659\\
329	-14.6535728638436\\
410	-15.0509054551711\\
510	-15.1474433364207\\
634	-16.0710536230435\\
789	-17.0747884045863\\
981	-17.124000490364\\
1221	-16.8953891542279\\
1519	-16.8151075825178\\
1889	-17.6257581989441\\
2350	-17.793775332005\\
2924	-19.8392344321366\\
3638	-18.532805300452\\
4526	-20.1129131491036\\
5630	-19.7336836590982\\
7004	-18.7068654285737\\
8714	-20.751436107845\\
10841	-21.3808871636708\\
13486	-21.289281626053\\
16778	-21.5054139314917\\
20873	-22.7495422127785\\
25968	-23.4650503975236\\
32306	-24.0368611756608\\
40191	-22.4255003256679\\
50000	-22.8186595298107\\
};
%\addlegendentry{RPCA, $\lambda=1/\sqrt{M}$}

\addplot [color=mycolor4, line width=1.0pt]
  table[row sep=crcr]{%
10	-5.06875476247908\\
12	-5.05739405471244\\
15	-4.56995013198082\\
19	-6.02499322634172\\
24	-6.18829668574758\\
30	-6.44227687264522\\
37	-7.58873095526091\\
46	-9.52644502401754\\
57	-10.5134651699859\\
71	-11.6050299615919\\
89	-12.6099123758194\\
110	-12.7894037125213\\
137	-13.2352038767998\\
171	-13.1046239386758\\
213	-15.4062994295801\\
265	-15.1280388077457\\
329	-15.013862621073\\
410	-15.6063203965172\\
510	-16.2161045851166\\
634	-15.1484104486753\\
789	-16.1522317214573\\
981	-17.2668688933758\\
1221	-17.9895582215644\\
1519	-18.5151594684598\\
1889	-19.3267182418018\\
2350	-18.893409870226\\
2924	-20.5150483363554\\
3638	-20.1459246870733\\
4526	-20.0449485866823\\
5630	-21.7704908361065\\
7004	-21.6338595764288\\
8714	-21.7424197826429\\
10841	-21.95950258223\\
13486	-22.2697210646778\\
16778	-22.649827196455\\
20873	-23.5648970091133\\
25968	-24.742905919077\\
32306	-23.8619501967235\\
40191	-24.9810435886319\\
50000	-24.9053281486199\\
};
%\addlegendentry{MCMC}

\end{axis}
\end{tikzpicture}%
		 \hfill\\[1cm]
		 
		 
		 %legend
		\definecolor{hald}{rgb}{0.00000,0.44700,0.74100}%
		\definecolor{rpca}{rgb}{0.85000,0.32500,0.09800}%
		\definecolor{mcmc}{rgb}{0.49400,0.18400,0.55600}%
		\tikz[baseline]{\draw[line width=1.0pt,color=hald] (0,.5ex)--++(.5,0) ;} DRec, 
		\tikz[baseline]{\draw[dashed,line width=1.0pt , color=rpca] (0,.5ex)--++(.5,0) ;} RPCA using $\lambda_{opt}$,
		\tikz[baseline]{\draw[line width=1.0pt , color=rpca] (0,.5ex)--++(.5,0) ;} RPCA using $\lambda=1/\sqrt{M}$,
		\tikz[baseline]{\draw[line width=1.0pt , color=mcmc] (0,.5ex)--++(.5,0) ;} MCMC\\
	\end{minipage}
	}
\end{frame} 
%Texte du diapo \cite{Solomaa1973} et \cite{Dijkstra1982}
%\vfill
% 
%{\tiny 
%\usebibitemtemplate{\color{black}\insertbiblabel} 
%\usebibliographyblocktemplate{\color{black}}{\color{black}}{\color{black}}{\color{black}} 
% 
%\begin{thebibliography}{} 
%\bibitem{Solomaa1973} 
%A.~Salomaa. 
%\newblock {\em Formal Languages}. 
%\newblock Academic Press, 1973. 
%\bibitem{Dijkstra1982} 
%E.~Dijkstra. 
%\newblock Smoothsort, an alternative for sorting in situ. 
%\newblock {\em Science of Computer Programming}, 1(3):223--233, 1982. 
%\end{thebibliography} }


\end{document}