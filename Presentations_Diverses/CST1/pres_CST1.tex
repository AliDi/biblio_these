\documentclass[10pt,xcolor=x11names,compress, notes=show]{beamer}% pour l'impression, tout n'apparait qu'une fois \documentclass[handout,12pt]{beamer}

%\usepackage[scaled]{helvet}
\usepackage[round]{natbib}
\usepackage[utf8x]{inputenc}
%\usepackage{ucs}
\usepackage[USenglish]{babel}
\usepackage{todonotes}
\usepackage{color}
%\usepackage{subfigure}
%\usepackage[]{geometry}
\usepackage{changepage}
\usepackage{pifont} %pour les symbole sympa \ding{nb}

% Pour Tikz
\usepackage{tikz}
\usetikzlibrary{calc}
%\usetikzlibrary{arrows,shapes,trees,positioning}  

%pour les plots matlab en tikz
\usepackage{pgfplots} 
\pgfplotsset{compat=newest}

% Pour les maths
\usepackage{bm}
\usepackage{amsmath,mathtools}
\usefonttheme[onlymath]{serif}
\usepackage{cancel} %pour barrer des math

% Pour la mise en forme
\usepackage[export]{adjustbox}
\usepackage{subcaption}
\usepackage{wrapfig}
\usepackage{pdfpages}
\setbeamertemplate{navigation symbols}{} 
\usepackage{array}
%\usepackage{palatino}
%\setbeamertemplate{caption}{\raggedright\insertcaption\par}
\usepackage{multicol}
\setlength{\columnsep}{0cm}
\usepackage[framemethod=TikZ]{mdframed}


%pour le theme
\usetheme{Alice}

%\useoutertheme[subsection=false]{miniframes} %%pour avoir le défilement en en-tête des diapos par section
%\setbeamercolor*{lower separation line head}{bg=DeepSkyBlue4} 

% Customisation 
\setlength{\fboxrule}{0.2pt}
\definecolor{green}{rgb}{0,0.5,0} 
\newcommand{\diag}[1]{\operatorname{diag}\left(#1\right)}
\newcommand{\tikzmark}[1]{\tikz[remember picture] \coordinate (#1) ++ (-3pt,6pt) {};}
\newcommand{\citeTransp}[1]{\color{fg!50} \citep{#1}}
\renewcommand\bibsection{\section[]{~}}
\usepackage{algorithm}
\usepackage{algorithmic}


%%% Page de titre
%======================
%\author{Alice Dinsenmeyer}
\institute{\small \underline{Encadrement} : \\ Jérôme Antoni (LVA)\\ Christophe Bailly (LMFA) \\ Quentin Leclère (LVA)\\[1em]
\underline{Référent} : \\ Laurent Maxit (LVA)\\[1em]
\underline{Représentant MEGA} : \\ Simon Chesné (LaMCoS)}
\title{Comité de suivi de thèse}
\subtitle{Alice Dinsenmeyer -- Fin de 1\textsuperscript{ère} année}
\date{}
\titlegraphic{}
\author{}


\begin{document}

%%%		Title
%======================
\begin{frame}[plain,t]
	\maketitle	
	\centering{ \vfill  \itshape \footnotesize 26 juin 2018}
\end{frame}

%%% Context
%======================
\newcommand{\done}[2]{\onslide<#1->{{\bfseries  \hfill $\triangleright$ #2}}}

\section*{Sujet}
\begin{frame}{\insertsectionhead}
	%\vspace{-0.3cm}
	%\linespread{1.5}
	\begin{center}
		 \hspace{-0.5cm} \bfseries Méthodes inverses pour la caractérisation de sources aéro-acoustiques
	\end{center}
	\begin{overlayarea}{\textwidth}{0.5\textwidth}
		%\linespread{1.5}
		\only<1-1>{
		\begin{itemize} \setlength\itemsep{1em}
	        		\item Financement : 50\% CeLyA + 50\% ADAPT \\ {\small (projet européen : LVA, LMFA, MicrodB, Airbus)}
	        		\item Contexte : Réduction du bruit des avions 
	        		\item Objectif : Localiser et  quantifier les sources de bruit de turbomachine et aérodynamique
	        		\begin{itemize}
        				\item[\ding{55}] fort bruit de CLT
        				\item[\ding{55}] sources large bande
			\end{itemize}
	        		\item Méthodes existantes : formation de voies et déconvolution	        		
			\begin{itemize}
			        \item[\ding{51}] flexibles, simple, rapide
			        \item[\ding{55}]   nécessite un bon modèle de source, sources corrélées, niveaux
			\end{itemize}
		\end{itemize}
		}
		
		\only<2-2>{
		Axes principaux de la thèse :  \\[1em]
	        	\begin{enumerate} \setlength\itemsep{1em}
        			\item Débruitage des mesures : $\bm{S}_{pp} = \bm{S}_{\text{acoustique}} + \bm{S_{\textsc{clt}}}$
        			\item Identification des sources (quantification/localisation)
        			\vspace{2em}
        			\begin{itemize}
        				\normalsize
        				\item[\ding{222}] Séquentiellement ou simultanément
        				\item[\ding{222}] Approche bayésienne
        				\item[\ding{222}]  Utilisation des connaissances/incertitudes a priori sur les sources et le bruit
			\end{itemize}

		\end{enumerate}

	 	}
 	\end{overlayarea}
 	
\end{frame}

\section*{Objectifs}
\begin{frame}{\only<1-2>{\insertsectionhead ~-- 1\textsuperscript{ère} année}\onslide<3->{Travail réalisé -- 1\textsuperscript{ère} année}}

	
	
	%\linespread{1.5}
 	
 	\only<3>{ \section*{Travail réalisé}}
 	\onslide<2->{
 	\begin{itemize} \setlength\itemsep{1em}
        		\item État de l'art et positionnement critique du sujet de thèse : 
        		\begin{itemize}
        			\item imagerie acoustique 	\done{3}{État de l'art et implémentation}
        			\item inférence bayésienne	\done{4}{Étude partielle}
        			\item physique des sources aéroacoustiques	\done{5}{Étude partielle}
        			\item débruitage de la matrice interspectrale	\done{6}{État de l'art \\ \hfill  $\triangleright$  Présentation à la Berlin Beamforming Conference}
		\end{itemize}
		\item Prendre en main les codes  de focalisation bayésienne (LVA)	\done{7}{Fait}
		\item Créer des cas tests pour la comparaison des méthodes	\done{8}{En cours}
	\end{itemize}
	}
 	
\end{frame}



\section*{Perspectives}
\begin{frame}{\insertsectionhead~--~2\textsuperscript{ème} année}
%\linespread{1.3}
	\begin{itemize} \setlength\itemsep{1em}
	        \item État de l’art sur les méthodes bayésiennes pour la localisation de sources
	        \item Extension des méthodes d’imagerie bayésienne existantes au contexte aéroacoustique
	        \item Élaboration de cas tests numériques
	        \item Applications sur des données expérimentales de laboratoire ou de contexte industriel (projet ADAPT)
	\end{itemize}
\end{frame}

\section*{Formation}
\begin{frame}{\insertsectionhead}
	\begin{itemize}
	        \item Formation scientifique (/20h) :
	        \begin{itemize}
	        		\item Cours d’aéroacoustique (16h)
	        		\item Cours Identification des systèmes et décomposition parcimonieuse des signaux (16h)
		\end{itemize}
	        \item Formation professionnalisante (/40h)
	        \begin{itemize}
	        		\item \underline{À faire en 2\textsuperscript{ème} année}
		\end{itemize}
		\item Séminaires (/6) : 
		\begin{itemize}
	        		\item 2 séminaires MEGA
	        		\item 2 séminaires non-MEGA
	        		\item Reste 2 à faire
		\end{itemize}
	\end{itemize}
\end{frame}

\section*{}










%Texte du diapo \cite{Solomaa1973} et \cite{Dijkstra1982}
%\vfill
% 
%{\tiny 
%\usebibitemtemplate{\color{black}\insertbiblabel} 
%\usebibliographyblocktemplate{\color{black}}{\color{black}}{\color{black}}{\color{black}} 
% 
%\begin{thebibliography}{} 
%\bibitem{Solomaa1973} 
%A.~Salomaa. 
%\newblock {\em Formal Languages}. 
%\newblock Academic Press, 1973. 
%\bibitem{Dijkstra1982} 
%E.~Dijkstra. 
%\newblock Smoothsort, an alternative for sorting in situ. 
%\newblock {\em Science of Computer Programming}, 1(3):223--233, 1982. 
%\end{thebibliography} }


\end{document}