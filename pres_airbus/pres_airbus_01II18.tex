\documentclass[10pt,xcolor=x11names,compress, notes=show]{beamer}% pour l'impression, tout n'apparait qu'une fois \documentclass[handout,12pt]{beamer}

%\documentclass[xcolor=x11names]{beamer}
%\usepackage[scaled]{helvet}
\usepackage[round]{natbib}
\usepackage[utf8x]{inputenc}
%\usepackage{ucs}
\usepackage[french]{babel}
\usepackage{todonotes}
\usepackage{tikz}
\usepackage{color}
%\usepackage{subfigure}
%\usepackage[]{geometry}
\usepackage{changepage}
\usetikzlibrary{calc}

\usepackage{bm}
\usepackage{pifont} %pour les symbole sympa \ding{nb}
\usepackage[export]{adjustbox}
\usepackage{subcaption}

\usepackage{pdfpages}
\setbeamertemplate{navigation symbols}{} 
%\usepackage{palatino}

%pour le theme
%\usetheme{CambridgeUS}

%\usetheme{Goettingen}
%\useinnertheme{default}
%\useoutertheme[subsection=false]{miniframes} %%pour avoir le défilement en en-tête des diapos par section
\setbeamertemplate{blocks}[rounded][shadow=true]
\setbeamercolor{block title}{fg=DeepSkyBlue4,bg=DeepSkyBlue4!10}
\setbeamercolor{block title alerted}{bg=DeepSkyBlue4!0} 
\setbeamercolor{block title example}{bg=DeepSkyBlue4!20}
%\setbeamercolor*{lower separation line head}{bg=DeepSkyBlue4} 

\setbeamerfont{title like}{shape=\scshape}
\setbeamercolor{frametitle}{fg=DeepSkyBlue4}
\setbeamercolor{title}{fg=DeepSkyBlue4}
\setbeamercolor{itemize item}{fg=black}
\setbeamercolor{itemize subitem}{fg=black}
\setbeamercolor{toc}{fg=DeepSkyBlue4}
\usepackage{amsmath,mathtools}
\usefonttheme[onlymath]{serif}
\usepackage{bm}

\setbeamertemplate{frametitle}{\vspace{0.5cm}\hspace{-0.9cm} \insertframetitle}
\setbeamerfont{frametitle}{size=\Large}

\setlength{\fboxrule}{1pt}

%\setbeamertemplate{bibliography item}[]

%couleur table des matières
\usepackage{hyperref}
\hypersetup{colorlinks=true, linkcolor=DeepSkyBlue4}

\setbeamertemplate{caption}{\raggedright\insertcaption\par}

%Mettre la section courante en titre de diapo (pour champ de titre non-vide)
%\addtobeamertemplate{frametitle}{\frametitle{\insertsubsectionhead}}{}

%\addtobeamertemplate{footline}{\hspace{11cm} \insertframenumber}
\setbeamertemplate{footline}[frame number]
%\newcommand{\tr}[1]{\prescript{t\hspace{-0.08cm}}{}{#1}}

\usepackage{wrapfig}


%page de titre
\author{J. {Antoni}, A. {Dinsenmeyer} and Q. {Leclère} \\ Laboratoire Vibrations Acoustique}

\title{Denoising of the CSM}
\subtitle{}
\date{February 2018}

%\titlegraphic{}


\begin{document}

\begin{frame}
	\maketitle
\end{frame}

\begin{frame}{Context}

\end{frame}

\begin{frame}{CSM properties}

$$\bm{S}_{p} = \frac{1}{N_s} \sum_i  \bm{p}_i\bm{p}_i'$$
\begin{itemize}
   	 \item Hermitian (conjugate symmetric)
   	 \item Positive semidefinite (nonnegative eigenvalues)
\end{itemize}~\\

$$\underbracket[0.5pt]{\bm{S}_p}_{\text{measured CSM}} =\underbracket[0.5pt]{ \bm{S}_a}_{\text{signal of interest}} +\underbracket[0.5pt]{\bm{S}_n}_{\text{unwanted noise}}$$

\begin{itemize}
     \item Signal CSM : one eigenvalue for one incoherent source
    \item Noise CSM : off-diagonal elements $\rightarrow 0$ with averaging
\end{itemize}
\end{frame}

\begin{frame}{Denoising algorithms}

\begin{block}{Diagonal reconstruction}
\begin{equation*}
    \text{maximize~} \sum_i \sigma_{n_i} \text{~~subject to~~} \bm{S}_{pp}-\operatorname{diag}(\bm{\sigma}_n)\geq 0
\end{equation*}
Solved with CVX Matlab toolbox
\todo[inline]{citation}
\end{block}

\begin{block}{Robust Principal Component Analysis}
\begin{equation*}
	\text{minimize~} \|\bm{{S}}_{a} \|_* + \lambda \| \bm{{S}}_{n} \|_1  \text{~~~~subject to~~~~}  \bm{{S}}_{aa} +  \bm{{S}}_{nn} = \bm{S}_{pp}
\end{equation*}
Solved with proximal gradient algorithm \todo[inline]{citation}
\end{block}

\begin{block}{Probabilistic Factorial Analysis}
\begin{equation*}
        \bm{p} = \bm{L}\operatorname{diag}(\bm{\alpha})\bm{C} +\operatorname{diag}(\bm{\sigma}^2)
\end{equation*}
\begin{equation*}
   \bm{L},\bm{C},\bm{\sigma}^2\sim \mathcal{N}(0,\bm{\Omega}_{L,C,\sigma}^2)~~~~\text{and}~~~~\bm{\alpha}\sim \mathcal{N}(\bm{\mu}_{\alpha},\bm{\Omega}_{\alpha}^2)
\end{equation*}
Solved with Gibbs sampling algorithm
\todo[inline]{citation}
\end{block}
\end{frame}

\begin{frame}{Results}

\end{frame}

\begin{frame}
 
%Texte du diapo \cite{Solomaa1973} et \cite{Dijkstra1982}
%\vfill
% 
%{\tiny 
%\usebibitemtemplate{\color{black}\insertbiblabel} 
%\usebibliographyblocktemplate{\color{black}}{\color{black}}{\color{black}}{\color{black}} 
% 
%\begin{thebibliography}{} 
%\bibitem{Solomaa1973} 
%A.~Salomaa. 
%\newblock {\em Formal Languages}. 
%\newblock Academic Press, 1973. 
%\bibitem{Dijkstra1982} 
%E.~Dijkstra. 
%\newblock Smoothsort, an alternative for sorting in situ. 
%\newblock {\em Science of Computer Programming}, 1(3):223--233, 1982. 
%\end{thebibliography} }
\end{frame}



\end{document}