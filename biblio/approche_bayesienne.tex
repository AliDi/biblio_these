\section{Approche bayésienne}

Rappel des estimation de sources et du paramètre de régularisation proposés par Antoni 2012.\\



Le problème peut être formulée à l'aide d'une approche probabiliste. Les inconnues du problème sont décrites  par une densité de probabilité. L'objectif de cette approche est double : \textit{i)} développer un formalisme généralisant les diverses méthodes développées pour chaque contexte d'imagerie ; \textit{ii)} prendre en compte au mieux les informations connues à l'avance sur les sources, même si elles sont incertaines puisque données par une densité de probabilité. Ce deuxième point est mis en œuvre par l'utilisation d'une régularisation.



la probabilité est la traduction d'un état de connaissance du système.


Principe général des probabilités bayésiennes : on choisit une distribution a priori décrivant la fonction à modéliser $[\bm{q}]$. Cette distribution est corrigée progressivement sous forme de fonction de vraisemblance à partir des observations $[\bm{p|q}]$. Prises ensemble, la distribution a priori et la fonction de vraisemblance permettent de construire la distribution a posteriori : $[\bm{q|p}] = [\bm{p|q}][\bm{q}]/[\bm{p}]$.

\subsection{Formulation probabiliste du problème direct}

Le problème direct revient à exprimer la pression $\bm{p}$ au niveau de l'antenne de microphones en fonction du champ source $\bm{q}$, du modèle de propagation acoustique $\bm{G}$, du bruit de mesure et des erreurs de modèle représentés par le vecteur $\bm{n}$ : 
\begin{equation}
\bm{p}=\bm{Gq}+\bm{n}
\end{equation}
Le champ source $q(\bm{r})$ peut est décomposé sur une base de $K$ fonctions spatiales $\phi_k(\bm{r})$ normalisées :
\begin{equation}
q(\bm{r}) = \bm{c} \cdot \bm{\phi}
\end{equation}

Les inconnues du problèmes sont donc les fonctions $\phi_k$, les coefficients $c_k$ qui dépendent des mesures et leur nombre $K$.\\

L'approche bayésienne propose de voir ces coefficients comme des variables aléatoires et d'étudier leur probabilité conditionnée aux mesures $[q(\bm{c},\bm{\phi})|\bm{p}]$. Si cette probabilité est élevée, ça signifie que les mesures expliquent précisément le champ source $q$. L'objectif est donc d'estimer ces variables de façon à ce qu'elles expliquent au mieux les mesures $\bm{p}$. Ces estimations de $\bm{\phi}$ et de $\bm{c}$ sont notées respectivement $\bm{\hat{\phi}}$ et $\bm{\hat{c}}$ telles que : 
\begin{equation}
	(\bm{\hat{c}},\bm{\hat{\phi}}) = \arg\max_{\bm{c},\bm{\phi}}[q(\bm{c},\bm{\phi})|\bm{p}]
\end{equation}
La loi de Bayes permet d'exprimer $[q(\bm{c},\bm{\phi})|\bm{p}]$ ainsi : 
\begin{equation}
\label{bayes}
[q(\bm{c},\bm{\phi})|\bm{p}]=\frac{[\bm{p}|q(\bm{c},\bm{\phi})][q(\bm{c},\bm{\phi})]}{[\bm{p}]}.
\end{equation}



On estime d'abord que le bruit $\bm{n}$  a une distribution gaussienne, et que sa moyenne est nulle ($\mathbb{E}\{\bm{n}\}=0$) et par conséquent 
\footnote{Modèle direct : $\bm{p}=\bm{Gq}+\bm{n}$ ; bruit centré : $\langle \bm{n}\rangle = \langle \bm{p}-\bm{Gq} \rangle = 0 $ ; la covariance du bruit est la covariance des mesures étant données les sources : 
\begin{align*}
\bm{C_n}& =  \mathbb{E}\{\bm{n}\bm{n}^*\} \\
		& =  \mathbb{E}\{(\bm{p}-\bm{Gq})(\bm{p}-\bm{Gq})^*\}  \\
		& =  \bm{C_{\bm{p}|\bm{q}}} 
\end{align*}
},
$[\bm{p}|q]$ suit également une distribution normale multivariée complexe. En introduisant la matrice de covariance $\mathbb{E}\{\bm{n}\bm{n}^*\}=\beta^{2}\bm{\Omega}_N$ ($\beta^2$ étant l'énergie moyenne du bruit, $\bm{\Omega}_N$ matrice connue a priori selon la nature du bruit) : 
\begin{align}
[\bm{p}|q,\beta^2] &=\mathcal{N}_c(\bm{G}\bm{q},\beta^2\bm{\Omega}_N) \\
	&= \frac{   1   }{ \pi^M \beta^{2M} |\bm{\Omega}_N| } \exp \left( -\frac{1}{\beta^2} ||   \bm{p}  -  \bm{Gq}  ||^2_{\bm{\Omega}_N} \right)
\end{align}

Une distribution gaussienne est également choisie pour la densité de probabilité des sources, de moyenne nulle et de variance $\mathbb{E}\{\bm{q}\bm{q}'\}=\alpha^2 \bm{\Omega}_q$ : 
\begin{align}
	[\bm{q}|\alpha^2] &= \mathcal{N}_c(\bm{0}, \alpha^2\bm{\Omega}_q)\\
	 &= \frac{  1  }{  \pi^K\alpha^{2K} |\bm{\Omega}_q|} \exp \left( -\frac{1}{\alpha^2}   ||\bm{q}||^2_{\bm{\Omega}_q} \right)
\end{align}

$\alpha^2$ et $\beta^2$ sont appelés les hyperparamètres.


\todo[inline]{ Notes :\\
vraisemblance : adéquation entre une distribution observée (sur échantillon) et la loi de proba qui décrit la population dont est issu l'échantillon 
fonction de vraisemblance : la vraisemblance varie  en fonction des paramètres de la loi choisie. Paramètre s'appelle généralement $\theta$. Sert donc a ajuster des observations à une loi.\\



\url{https://en.wikipedia.org/wiki/Bayesian_interpretation_of_kernel_regularization} : \\

Dans le contexte de régression, la fonction de vraisemblance ($[p|q]$) sont souvent supposé suivre une distribution gaussienne, car ``corrompues'' par du bruit gaussien. Les observation sont supposées indépendantes et identiquement distribuées, ce qui fait qu'il est possible de factoriser la fonction de vraisemblance sur chaque point de mesure.
}

\subsection{Estimation des hyperparamètres et des sources}


\subsubsection{Choix de la fonction coût}

Il existe trois principales approches pour poser le problème d'optimisation et en définir une fonction objectif \footnote{détails des calculs dans le cours de E. Thiébaut et C. Pichon : \url{https://cral.univ-lyon1.fr/labo/perso/eric.thiebaut/downloads/documents/cargese-2006-thiebaut.pdf}} : 

\paragraph{\tbullet le maximum de vraisemblance}, pour lequel on cherche la solution 
	\begin{equation}
		\bm{q}_{ML} = \arg\max_{\bm{q}} [ \bm{p}|\bm{q} ]=\arg\min_q f_{data},
	\end{equation}	
	 avec 
	\begin{eqnarray}
		f_{data} &=&-\log[\bm{p}|\bm{q}] + \text{cst}\\
				 &=& [\bm{p}-\bm{Gq}]'  \bm{C}_{\bm{p}|\bm{q}}^{-1}  [\bm{p}-\bm{Gq}]
	\end{eqnarray}
	en considérant que $[ \bm{p}|\bm{q} ]$ a une répartition gaussienne, de covariance $\bm{C}_{\bm{p}|\bm{q}}$ . L'estimateur obtenu est le même que celui donné par la méthode des moindres carrés.\\
	 	
\paragraph{\tbullet le maximum a posteriori} (MAP), qui cherche à maximiser la densité des sources a posteriori : 
	\begin{eqnarray}
		\bm{q}_{MAP} &=& \arg\max_{\bm{q}}[ \bm{q} |  \bm{p}  ]\\
					& =& \arg\max_{\bm{q}}([\bm{p} | \bm{q}][\bm{q}])\\
					&=& \arg\min_{\bm{q}} ( \underbrace{-\log[\bm{p}|\bm{q}]}_{f_{data}} \underbrace{- \log[\bm{q}]}_{f_{prior}})
	\end{eqnarray}
	La méthode MAP propose donc d'ajouter à au maximum de vraisemblance un terme de régularisation donné par $f_{prior}$. Considérant que les sources ont une densité de probabilité gaussienne, le terme de régularisation s'écrit : 
	\begin{equation}
		f_{prior}= (\bm{q}- \bm{\bar{q}})' \bm{C_q}^{-1}(\bm{q}- \bm{\bar{q}})
	\end{equation}\\	
	
\paragraph{\tbullet le minimum de variance} \footnote{\textit{minimum mean square error}}, dont le critère est de minimiser l'espérance de l'erreur quadratique : 
	\begin{equation}
		\bm{q}_{MV} = \arg\min_{\tilde{q}} \mathbb{E}\{||\bm{q}-\bm{\tilde{q}}||^2\}
	\end{equation}	
	Cette solution généralise le filtre de Wiener et est la même que la solution MAP. \\


En faisant un analyse de la propagation des erreurs pour chaque méthode, on peut montrer que le maximum de vraisemblance en propage davantage.\\

\subsubsection{Estimateur MAP}
La résolution du problème inverse se fait donc par l'estimation de la probabilité $[\bm{q}|\bm{p},\alpha^2,\beta^2]$. Le vecteur $\bm{q}$ est ainsi approché en observant sa valeur la plus probable d'après les données, le  ``Maximum a posteriori'' (MAP) : 
\begin{align}
\bm{\tilde{q}}_{MAP} &= \arg\max([\bm{q}|\bm{p},\alpha^2,\beta^2])\\
					 &= \arg\max\left( [\bm{p}|\bm{q}][\bm{q}]  \right) ~~~\text{d'après \ref{bayes}}
\end{align}
puisque $[\bm{p}]$ est une constante du problème.

En prenant le logarithme négatif de la quantité à maximiser, on peut définir une fonction coût à minimiser : 
\begin{equation}
	J(\bm{c},\bm{\phi}) = - \ln[q|\bm{p}] = -\ln[\bm{p}|q]-\ln[q]
\end{equation}

D'après les lois normales choisies, 
\begin{equation}
J=M\ln(\beta^2) + K\ln(\alpha^2) +\beta^{-2}||\bm{p}-\bm{Hc}||^2_{\Omega_n}+\alpha^{-2}||\bm{q}||^2_{\Omega_c}
\end{equation}
\todo[inline]{revoir les notations ci-dessus}
Finalement, on ne conservant que les termes qui dépendent de $c$, on retrouve la formulation des moindres carrés : 
\begin{equation}
 J = (p'-cH)\Omega^{-1}_n(p-Hc) + \nu^2c'\Omega^-1_c c
\end{equation}
avec pour terme de régularisation $\nu = \beta^2/\alpha^2$.





\subsubsection{MCMC}
algo utilisé par Charles : Metropolis-Hasting


\subsection{Confiance accordée à la reconstruction}

	
