\section{Ajout d'une étape de déconvolution\label{deconvolution}}

La distribution de sources obtenue obtenue par une méthode d'imagerie peut être vue comme la convolution entre la distribution de sources et la fonction d'étalement du point (PSF : point spread function). La PSF est comparable à une réponse impulsionnelle du système d'imagerie. En formation de voies, la PSF est souvent connue \citep{Bahr2011a}: elle est composée d'un lobe principal et de lobes secondaires. \\

Ces lobes diminuent notamment le pouvoir de séparation des sources, surtout à basses fréquences ou si les sources sont proches ou encore multipolaires. Si la PSF est connue, on peut, en principe, déconvoluer la distribution  de source calculée afin de réduire l'intrusion des lobes secondaires.\\

On distingue 2 types de lobes secondaires : ceux liés à l'espacement entre les microphones et ceux généré par la forme générale de l'antenne (le fait qu'elle soit d'une surface finie). Ces derniers peuvent être corrigés en appliquant une fenêtre d'appodisation diminuant la sensibilité des microphones situés sur les bords de l'antenne.\\


La PSF des méthodes inverses est difficile à estimer car elle dépend des données de mesure. C'est pourquoi les méthodes de déconvolution y sont moins appliquées. En revanche, la PSF du beamforming est bien connue et la résolution de cette méthode est souvent améliorée par une étape de déconvolution.\\

\subsection{PSF du beamforming standard}

Le beamforming standard permet d'estimer les sources $\bm{\tilde{q}}$ ainsi (cf paragraphe~\ref{bf_standard}) : 
\begin{equation}
	\bm{\tilde{q}} = \bm{W}\bm{p},
\end{equation}

ou bien, en terme d'énergie : 

\begin{align}\label{bf_psf}
	~ & \bm{\tilde{S}_{qq}} = \bm{W}' \bm{S_{pp}} \bm{W}  \\
	\text{avec, ~~~~} & \bm{S_{pp}} = \bm{G S_{qq}G}'
\end{align}

Comme les sources sont supposées décorrélées, on peut calculer avec uniquement les diagonales de  $\bm{\tilde{S}_{qq}}=\text{diag}(b_1, ... , b_j...,b_N)$ et $\bm{S_{qq}}=\text{diag}(q_1, ... , q_j...,q_N)$.\\

On a alors : 
\begin{equation}
	b_j = \sum_k A_{jk}q_k
\end{equation}
avec  $\bm{A}$ la PSF associée à chaque point source. D'après~\ref{bf_psf}, et en rappelant que $\bm{G}=[\bm{g}_1,~...~,\bm{g}_N]$ et $\bm{W}=[\bm{w}_1,~...~,\bm{w}_M]^T$, on a : 
\begin{equation}
	A_{jk}=|\bm{w}_j'\bm{g}_k|^2
\end{equation}
\todo[inline]{Rapport $1/M^2$ ?}

La déconvolution consiste théoriquement à inverser la matrice $\bm{A}$ afin de retrouver les vraies valeurs de sources. En pratique, il est nécessaire d'avoir recours à une inversion avec contrainte de positivité sur la solution ($\bm{A}$ étant de grande dimension). Ce problème est généralement suffisamment bien posé pour qu'aucune régularisation supplémentaire ne soit requise.

\todo[inline,color=green!10]{Notations inspirée de \url{http://www.bebec.eu/Downloads/BeBeC2014/Papers/BeBeC-2014-02.pdf} qui donne aussi g et w en présence d'un écoulement uniforme.}


\subsubsection{DAMAS}
%article en francais : http://www.conforg.fr/cfa2014/cdrom/data/articles/000206.pdf
 \cite{Brooks2006} proposent l'algorithme DAMAS (deconvolution approach for the mapping of acoustic sources) pour résoudre le problème linéaire de déconvolution : 
\begin{align}
	\Leftrightarrow~~~~ & b_j=\sum_k A_{jk}q_k
\end{align}
Les sources sont supposées incohérentes (i.e. distribution de sources indépendantes statistiquement).
Cette relation peut être décomposée, de manière analogue à la méthode de Gauss-Seidel : 
\begin{align}
	&b_j= \sum_{k=1}^{j-1}A_{jk}q_k + A_{jj}q_j + \sum_{k=j+1}^{N}A_{jk}q_k\\
	\Leftrightarrow~~~~~& q_j = \frac{1}{A_{jj}}\left( b_j - \sum_{k=1}^{j-1}A_{jk}q_k -  \sum_{k=j+1}^{N}A_{jk}q_k   \right)
\end{align}
avec $N$ le nombre total de points de l'image.
Cependant, les $q_k$ sont inconnus. Ils sont calculés de manière itérative : pour $k>j$, les $q_k$ sont calculés à partir de l'itération précédente et pour $k<j$, les sources sont données à l'itération courante par la relation : 

\begin{equation}
	q_j^n= \frac{1}{A_{jj}}\left(b_j -\sum_{k=1}^{j-1}A_{jk}q_k^{n} - \sum_{k=j+1}^{N}A_{jk}q_k^{n-1} \right)
\end{equation}
L'incrémentation se fait donc sur $j$, la position des sources déconvoluées et le nombre d'itération $n$ est au choix de l'utilisateur. Pour l'initialisation, on peut choisir $q_j=0$ ou bien $q_j=b_j$, ce qui fera une différence sur la vitesse de convergence.\\
Pour forcer la convergence, $q$ étant une valeur énergétique, elle est mise à zéro si une valeur est calculée négative.\\
Cette déconvolution permet de supprimer en partie les lobes secondaires mais présente le principal inconvénient d'être lent.\\

Il existe de nombreuse extension de cette méthode de déconvolution \citep{Dougherty2005}. Par exemple, DAMAS2 considère que la convolution avec la PSF est invariante par translation. Ainsi, la convolution dans le domaine spatial est remplacée par un produit dans le domaine des nombres d'onde. Cette formulation a pour effet d'accélérer la procédure de déconvolution. Elle propose aussi l'ajout d'une régularisation par un filtre passe-bas. Ce choix de PSF peut être une bonne approximation si les sources sont suffisamment éloignées de l'antenne. DAMAS-C prend en compte une définition cohérente des sources.%brooks 2006



 
\subsubsection{Spectral Estimation Model (SEM)}
Correction de différence entre deux matrices de covariance minimisée avec un gradient conjugué \citep{Blacodon2003}\\
Equivalent à un filtre de Wiener ?
Contrainte de positivité sur la solution de source difficile à appliquer ?



\subsubsection{Non-negative least squares (NNLS)}
L'approche NNLS est de minimiser l'erreur au sens des moindres carrés entre $\bm{b}$ et $\bm{Aq}$, en imposant que $\bm{q}$ reste non-négatif : 
\begin{eqnarray}
\min_{\bm{q}}\left(||\bm{\tilde{S_{qq}}}-\bm{A}\bm{S_{qq}}||^2 \right)\\
\text{~~~~sous la contrainte~~~~}\bm{S_{qq}}\geq 0
\end{eqnarray}
\todo[inline,color=green!10]{Possibilité d'utiliser la fonction Octave lsqnonneg}

De manière similaire à DAMAS2, la version FFT-NNLS remplace la convolution par une multiplication dans le domaine des nombres d'onde pour accélérer les calculs.

\subsubsection{Déconvolution avec contrainte de parcimonie}
Les algorithme ci après font l'hypothèse que le vecteur source est parcimonieux

\paragraph{CLEAN}
Pour réduire les lobes secondaires produits par les méthodes d'imagerie, \citep{Hogbom1974} propose de traiter les données (issues des radiotéléscopes) de manière itérative à l'aide d'un algorithme CLEAN.\\
% L'algorithme CLEAN repose sur 3 principales hypothèses : les sources sont décorrélées ($\bm{S_{pp}}=\sum_k \bm{p}_k \bm{p}k$), il n'y a pas de décorrélation d'un microphone à un autre et il n'y a pas de bruit incohérent.\\
Le principe de l'algorithme est le suivant : on extrait la plus grande valeur du champ source issu du beamforming, on la note comme un point source, on lui retire un petit gain convolué avec la fonction d'étalement, et on réitère jusqu'à ce que la plus grande valeur atteigne un seuil. Le pseudo-code correspondant se trouve en figure~\ref{clean_hogbom}.\\


\cite{Sijtsma2007} propose une version de cet algorithme pour lequel, à chaque itération, ce sont les données qui sont nettoyées et l'image des sources est recalculée à chaque fois. Cette version permet notamment d'appliquer une pondération sur les données, comme une suppression de la diagonale pour réduire l'effet du bruit aérodynamique. Cette méthode est de type "matching pursuit" (voir le paragraphe sur l'optimisation parcimonieuse) : à chaque itération, le vecteur de pointages correspondant à une localisation de sources est écarté et le signal est projeté dans le nouvel espace vectoriel.\\

\begin{figure}[!h]
	\centering
	\fbox{
	\begin{minipage}{0.45\textwidth}	
		Partant de l'image des sources issue du beamforming $dirty$, calcule l'image déconvoluée $clean$.\\
		\begin{algorithmic}
			\STATE $gain = $ gain loop
			\STATE $clean = $zeros(shape($dirty$))
			\STATE $res=dirty$			
			\WHILE{$i < niter$ \AND max(abs($res$))< $thresh$}
			\STATE \textcolor{orange}{1. Search for the peak location in $dirty$}
			\STATE $rmax$ = coordonnées de max($res$)
			\STATE $mval = res[rmax]\times gain$
			\STATE \textcolor{orange}{2. Update $clean$}
			\STATE $clean[rmax]$ += $mval$
			\STATE \textcolor{orange}{3. Subtract the appropriately scaled PSF from the dirty map.}
			\STATE calcul des coordonnées des bords de la psf centrée sur $rmax$
			\STATE $res$[centré sur $rymax$]	-=  $psf$[centrée sur $rmax$] $\times mval$
			\STATE $i$ += $1$
			\ENDWHILE			
		\end{algorithmic}
	\end{minipage}
	}
	\caption{Pseudo-code de CLEAN de \cite{Hogbom1974}. \label{clean_hogbom}}
\end{figure}


\begin{figure}[!h]
	\centering
	\fbox{
	\begin{minipage}{0.45\textwidth}	
		Partant de la matrice interspectrale des signaux microphoniques $Spp$, calcule image beamforming $dirty$, puis l'image déconvoluée $clean$, puis la matrice $Sclean$ induite. \\
		\begin{algorithmic}
			\STATE $gain = $ gain loop
			\STATE $clean = $zeros($M,N$)
			\FOR{$i$ in $niter$}
				\STATE \textcolor{orange}{1. Obtain $dirty$ by beamforming}	
				\FOR{$n=1:N$}
					\STATE $dirty[n]=W'[n,:]\times Spp \times  W[n,:]$
				\ENDFOR 
				\STATE \textcolor{orange}{2. Search for the peak location in $dirty$}
				\STATE $rmax$ = coordonnées de max($dirty$)
				\STATE $mval=dirty[rmax]\times gain$
				\STATE \textcolor{orange}{3. Steering vector to location of peak}	
				\STATE $wpeak = W[rmax,:]$			
				\STATE \textcolor{orange}{4. Update $clean$}
				\STATE $clean[rmax]$ += $mval$
				\STATE \textcolor{orange}{5. Calculate $Sclean$ induced}
				\STATE $Spp$ -= $mval \times wpeak \times wpeak'$	
				\STATE \textcolor{orange}{6. Trimm $Spp$}
				\STATE $Spp$[logical(eye(N))]=0	
				\STATE \textcolor{orange}{Stop criterium : Frobenius norm of $Spp$ must always decrease}
				\STATE $normSpp$=normFrobenius($Spp$)
				\IF{$normSpp>normSclean$}
				\STATE stop
				\ENDIF	
				\STATE $normSclean=normSpp$
			\ENDFOR	
		\end{algorithmic}
	\end{minipage}
	}
	\caption{Pseudo-code de CLEAN-psf de \cite{Sijtsma2007}.}
\end{figure}
\todo[inline]{Pourquoi supprimer aussi la diagonale de $wpeak \times wpeak'$ ?}

 
\todo[inline,color=green!10]{algo Hogbom en python : \url{http://www.mrao.cam.ac.uk/~bn204/alma/python-clean.html}\\ algo Sijtsma en matlab : \url{https://github.com/jorgengrythe/beamforming/tree/master/algorithm}}




%\cite{Dougherty1998} présente 3 façons de réduire les lobes secondaires liés au positionnement des michrophones sur l'antenne : \\
%-CSM weighting : réduire le poids des termes de la CSM (cross-spectral matrix) correspondant aux paires de microphones dont l'espacement fait qu'ils apportent une information redondante (par exemple 2 microphones sur la même branche d'une antenne en croix)\\
%-robust adaptative beamforming = minimum variance algorithm = CAPON\\
%-CLEAN algorithme.\\
%De ces trois méthodes, CLEAN ressort comme étant la plus efficace



\paragraph{Sparcity constrained DAMAS (SC-DAMAS)}
\cite{Yardibi2008}

type basis pursuit

\paragraph{bayésien}


Comme l'étape de déconvolution revient à résoudre un problème inverse où toutes les sources sont traitées simultanément, les méthodes inverses décrites dans la section~\ref{methodes_inverses} peuvent être utilisées.


\subsubsection{Prise en compte de la cohérence des sources}
\paragraph{CLEAN-SC}
\cite{Sijtsma2007} propose une version de CLEAN qui prend en compte la cohérence des sources. Le principe est le suivant : Le maximum  $S_{max}$ de la dirty map est trouvé au point $\bm{r}_{max}$.  Les composantes spatiales corrélées avec ce maximum sont sélectionnées puis soustraites à la dirty map.\\

On veut trouver la matrice interspectrale $\bm{G}$ qui contient toutes les contributions des sources corrélées avec la source en $\bm{r}_{max}$. Ainsi, la projection de $\bm{G}$ sur le vecteur $\bm{w}_max$ est égale à la projection des données $\bm{S_{pp}}$ sur ce même vecteur $\bm{w}_max$ : \\
\begin{equation}\label{Gw=Dw}
	\bm{G}\bm{w}_max= \bm{D} \bm{w}_max
\end{equation}
\\

Il faut donc construire un vecteur de propagation $\bm{h}$ traduisant la cohérence entre $S_{max}$ et les autres sources, de manière à ce que 
\begin{equation}
	\bm{G} = S_{max} \bm{hh}'
\end{equation}
soit solution de l'équation~\ref{Gw=Dw}.
Cette solution n'est pas unique. On peut exprimer $\bm{h}$ de la façon suivante : 
\begin{equation}
	\bm{h}=\frac{1}{\bm{h}'\bm{w}_{max}}  \frac{\bm{S_{pp}}\bm{w}_{max}}{S_{max}}
\end{equation}
En réinjectant cette expression dans~\ref{Gw=Dw} et en rappelant que $S_{max} = \bm{w}_{max}\bm{S_{pp}}\bm{w}_{max}'$, on trouve que le scalaire $\bm{h}'\bm{w}_{max}$ doit être égal à $1$.\\

Si on veut utiliser un $\bm{G}$ à diagonale nulle, on cherche une expression de $\bm{h}$ tel que : 
\begin{equation}
	\bar{\bm{G}} = S_{max} (\bm{hh}'-\bm{H}),
\end{equation}
avec $\bm{H}$ contenant les éléments diagonaux de $\bm{hh}'$. En isolant $\bm{h}$ dans l'équation (Gwmax=Dwmax), on a : 
\begin{equation}
	\bm{h}=\frac{1}{\bm{h}'\bm{w}_{max}}  \left( \frac{\bm{S_{pp}}\bm{w}_{max}}{S_{max}} + \bm{H}\bm{w}_{max} \right)
\end{equation}

En réinjectant cette expression dans (Gwmax=Dwmax)~\ref{Gw=Dw}, on trouve que le scalaire $\bm{h}'\bm{w}_{max}$ doit être égal à $\sqrt{1+\bm{w}_{max}\bm{H}\bm{w}_{max}'}$.




Cette méthode offre une meilleure résolution que CLEAN lorsque les sources sont corrélées.


\paragraph{DAMAS-C}





référence pour chaque méthode dans bahr2011 : \\
-damas2\\

-fft-nnls\\

-clean-sc\\

-cmf et son extension aux sources cohérentes (CFM-C)\\

-lore\\

-macs\\


~\\  : DAMAS, CLEAN-SC, TIDY

lire review : Sparsity constrained deconvolution approaches for acoustic
source mapping


Ces méthodes supposent de bien connaître son modèle de sources pour avoir une bonne PSF. Elles soont très utilisées dans le domaine de domaine de l'aéroacoustique (\cite{Fleury2012} propose une correction pour les sources en déplacement, par exemple).

