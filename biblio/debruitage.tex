\chapter{Séparation des composantes du bruit}
%===============================================

\section{Extraction du bruit de mesure}
Le bruit de mesure comprend principalement : le bruit ambient, le bruit électronique et le bruit aérodynamique.\\


Le bruit aérodynamique a des propriétés qui peuvent permettre de l'extraire des signaux de mesure : \\
-il est stationnaire et décorrélé des sources,\\
-sa longueur de corrélation spatiale est courte (à comparer avec l'espacement des micros) \\
-il ne génère pas de bruit acoustique (à discuter)\\
-son contenu spectral est connu (Empirical spectral model of surface pressure fluctuations ?) : large-bande et énergie équi-répartie sur les fréquences.\\

Le champ acoustique a, au contraire, une longueur de corrélation spatiale plus importante.\\

Finalement, la matrice d'autocorrélation du signal total peut s'écrire comme étant la somme des matrices d'autocorrélation des composantes acoustique et turbulente du signal. \\

acoustique : matrice à rang réduit si  nompbre réduit de sources\\
turbulence : matrice diagonale si on suppose qu'il y a une décorrélation totale entre les micros (ou une physique proche de la diagonale (décroissance exponentielle orthotrope, par exemple))\\

\begin{equation}
\bm{S_{yy}} = \bm{S_{xx}}' + \bm{S_{nn}}
\end{equation}
L'identification de ces matrice s'appelle "Structured Covariance Estimation problem".\todo[inline]{état de l'art}
$\bm{B}=Diag(\sigma^{2})$

Un filtrage dans le domaine des nombres d'ondes nécessite un grand nombre de microphone pour être fiable. De plus, il fait l'hypothèse que les parties acoustique et aérodynamique sont disjointes dans ce domaine, ce qui n'est vrai que pour $M<0.8$. D'autres stratégies de débruitage doivent donc être mises en place.

\subsection{Suppression des éléments diagonaux}
Si les signausx sont stationnaires et moyennés, le bruit incohérent devrait impacter principalement la diagonale de la CSM. Il est fréquent de supprimer ces éléments diagonaux. Cette opération a pour effet de rendre la CSM singulière et en y appliquant les méthode de beamforming, les niveaux de sources sont sous-estimés et peuvent même être négatifs.\\
Certaines méthodes comme le Functionnal BF  ou l'holographie fonctionne suuportent très mal la suppression de diagonale.

\subsection{Reconstruction de la diagonale de la CSM}
Ce type de méthode propose de résoudre un problème d'optimisation : minimiser la somme des éléments diagonaux de la CSM, sous la condition que la CSM reste semi-définie positive : 
\begin{equation}
    \arg\max_{\bm{d}}\sum_i d_i~~~~~\text{s. c.}~~~~~\bm{S_y}-\bm{d}\geq 0
\end{equation}

Pour cela, différents algorithmes d'opitmisation peuvent être employé.\\
Le pseudo-code ci-dessous est un exemple de solveur utilisant les outils de programmation convexe de Michael Grant et Stephen Boyd : CVX: Matlab software for disciplined convex programming, version 2.0 beta.\url{http://cvxr.com/cvx}, September 2013.

\begin{figure}[!h]
	\centering
	\fbox{
	\begin{minipage}{0.45\textwidth}	
		\begin{algorithmic}
			\STATE cvx\_begin
				\STATE variable d(M)
				\STATE  minimize( sum(d) )
				\STATE subject to
					\STATE lambda\_min(CSM $+$ diag(d))$>=$ 0
			\STATE cvx\_end
		\end{algorithmic}
	\end{minipage}
	}
	\caption{Exemple de code pour la reconstruction de diagonale}
\end{figure}
Cet algorithme s'arrête lorsque la plus petite valeur propre de la CSM modifiée atteint zéro. L'erreur de reconstruction ne va donc pas dépendre du niveau de bruit mais du spectre aux valeurs propres de la CSM. \\
~\\

\cite{dougherty2016} résoud le même problème sous forme d'une succession de problèmes d'optimisation linéaire.


\paragraph{\tbullet Alternating projections}
Cet algorithme permet de trouver l'intersection  (ou la plus petite distance) entre deux ensembles. Soit $E_1$ et $E_2$ ces deux ensembles. Le principe est de calculer itérativement le résultat $y^{(k)}$ de la projection de $x^{(k)}$ sur l'ensemble $E_1$, puis le résultat $x^{(k+1)}$ de la projection de $y^{(k)}$ sur l'ensemble $E_2$. Si les deux ensembles sont convexes, la convergence est linéaire.
Cyclic projections généralise AP à un nombre d'ensembles supérieur à 2.\\
Dans le cas du débruitage, les ensembles sont : l'ensemble des matrices semidéfinies positives, ensemble qui prend en compte la structure du bruit (éléments diagonaux).

\subsection{Exploitation de la cohérence du champ acoustique}
\cite{finez:hal-01276687} proposent une autre méthode de reconstruction, fondée sur l'hypothèse que le champ acoustique de la CSM est parfaitement cohérent : 
\begin{equation}
\frac{|{\bm{S_{pp}}}_{ij}|^2}{{\bm{S_{pp}}}_{ii}{\bm{S_{pp}}}_{jj}} =1
\end{equation}
La diagonale est alors calculée à partir de cette expression pour chaque couple $ij$. Cette méthode ne modifie pas l'interspectre. L'hypothèse de cohérence est difficile à vérifier en environnement très bruyant, notamment quand le bruit a des longueurs de corrélation supérieur à l'espacement inter-microhonique.\\


\subsection{Idendification en sous espaces}
Analyse en composantes principales de la CSM : 
\paragraph{\tbullet MUSIC} L'algorithme MUltiple Signal Classification (MUSIC, \cite{Schmidt1986}) propose une décomposition en valeurs propres de la matrice interspectrale $\bm{S_{pp}}$ pour la décomposer en 2 sous-espaces, l’un associé au signal et l’autre au bruit, afin de diminuer la contribution énergétique du bruit.  Dans l'équation~\ref{Gii}, $\bm{S_{pp}}$ est remplacé par les composantes correspondant au sous-espace bruit. Ainsi, ce nouvel estimateur sera maximal lorsque le processeur pointe vers une source, puisque les éléments du dénominateur seront décorrélés. Cet estimateur ne correspond alors plus à une densité spectrale des sources mais seulement à un indicateur de présence au point $i$. Cette méthode nécessite que le RSB soit suffisament bon et que la CSM ne soit pas de rang plein.\\


\paragraph{\tbullet  Proper orthogonal decomposition}

\subsection{Classical Principal Component Analysis (PCA)}
Méthode très utilisée en analyse, compression et visualisation de donnée (. Repose sur l'idée que dans une matrice de très grande dimension, l'information  se trouve dans un sous-espace de dimensions beaucoup plus petites. Si $Y$ est une matrice de dimension $m \times n$, elle eput alors s'écrire $S + N$ avec $S$ une matrice de rang $r$ très petit devant $m,n$ et les éléments de $N$ sont des variables gaussiennes. PCA propose donc de résoudre le problème suivant : 

\begin{equation}
    \min_{S,N} || S ||_F, ~~~\text{s. c.~~~} \rank(S) \le r ,~~Y=S+N.
\end{equation}

Ce problème peut être résolu par une décomposition en valeurs singulières de $Y$. $S$ est alors le résultat de la projection de $Y$ sur ses $r$ premières valeurs singulières (de gauche).
Cependant, cette méthode est mise en échec quand $N$ est de forte amplitude devant $S$. \\
Une autre formulation tente de remédier à ce problème : Robust PCA.

\subsection{Robust PCA (RPCA)}
Références et codes par l'université d'Illinois : \url{http://perception.csl.illinois.edu/matrix-rank/}//

La version dite robuste de PCA permet d'estimer $A$ en présence d'un fort bruit $N$.

\paragraph{\tbullet Méthodes non-convexes}


\paragraph{\tbullet Relaxation convexe}

Considérant que la matrice de bruit $N$  est de forte amplitude, mais parcimonieuse, on peut résoudre le problème suivant \citep{Wright2009a} : 
\begin{equation}
    \min_{S,N} ||S||_* + \lambda|N|_1,~~~~\text{s. c.~~~}Y=S+N.
\end{equation}
$\lambda$ est un paramètre de pondération.
Ce problème peut être résolu par les algorithmes classiques d'optimisation convexe .
 \cite{Wright2009a} utilisent un algorithme de seuillage itératif qui converge lentement. Depuis, de nombreux algortihmes ont été utilisés pour résoudre ce problème :  Augmented Lagrange multipllier,  Accelerated Proximal Gradient,  Dual Method, Singular Value Thresholding, Alternating Direction Method...\\
 
 Note : Ce problème est étroitement lié au problème de \textit{matrix completion}.	

Les méthodes de résolution du problème RPCA citées ci-dessus nécessitent d'ajuster $\lambda$  finement. La résolution de ce problème par une approche bayésienne permet de s'affranchir du choix de $\lambda$ (ref : X. Ding, L. He, and L. Carin, “Bayesian robust principal component  analysis,”). Bacadan (Sparse Bayesian Methods for Low-Rank Matrix Estimation) propose une méthode bayésienne qui permet, en plus, de ne pas choisir le rang de $S$. Dans cette méthode, la matrice $Y$ est décomposée comme suit : 

\begin{equation}
    S=AB^T = USV^T = \left(  US^{1/2} \right) \left(  S^{1/2}V^T \right)
\end{equation}
$S$ étant une matrice de rang réduit $r$. Le problème consiste alors à résoudre : 

\begin{equation}
    \min_{A,B} ||A||_{F}^{2}  + ||B||_{F}^{2},~~~~\text{s. c.~~~} ||Y-S-N||^2_{F} < \epsilon
\end{equation}
On peut montrer que  $\min_{A,B} ||A||_{F}^{2}  + ||B||_{F}^{2}$ est équivalent à $\min ||S||_*$ \footnote{pour la démo, voir Recht 201 : Guaranteed minimum-rank solutions of linear matrix equations via nuclear norm minimization}



\subsection{Approche statistique}

\paragraph{\tbullet Probabilistic PCA}
\cite{Tipping1999} proposent une approche probabiliste de la PCA (PPCA). Ils montrent que les axes principaux d'un ensemble peuvent être déterminés par une estimation de maximum de vraisemblance. Le principal sous-espace est déterminé par un algorithme EM.

\paragraph{\tbullet Analyse factorielle}
Méthode très proche de PPCA. La différence est que le bruit recherché n'est pas une constante, mais une matrice diagonale (dont les éléments non-nuls ne sont pas constants).



\subsection{Utilisation d'une mesure de bruit référente}
Différentes techniques sont basées sur une mesure de bruit de fond préliminaire pour le débruitage de la CSM : 
\begin{itemize}
	\item soustraction de la mesure de bruit de fond au spectre signal+bruit.
	\item Blacodon : Spectral Estimation Method With Additive Noise (SEMWAN)
\end{itemize}

\cite{Bulte2007} propose un décomposition en sous-espaces signal et bruit (nécessite une mesure de bruit) basée sur une une décomposition en valeurs singulires généralisée de la CSM.\\
Empirical Mode Decomposition\\

\todo[inline]{
- Rejection of flow noise using a coherence function method , Chung\\
-Biblio thèse PARISOT-DUPUIS (holographie soufflerie)
-Chung
}








\subsection{Méthodes expérimentales}

\paragraph{\tbullet Mesures vibratoires} L'écoulement perturbe la couche limite au niveau de l'antenne de microphones, ce qui génère un fort bruit aérodynamique. La mesure de ce bruit peut être fortement réduite en captant le champ acoustique à l'aide d'une antenne d'accéléromètres fixés à une plaque fine. Seuls les bas nombres d'onde, correspondant à la partie acoustique du champ d'onde sont alors mesurés. (Acoustic beamforming through a thin plate using vibration measurements  +   Design and Experimental Validation of an Array of Accelerometers for In-flow Acoustic Beamforming Applications) \\

Autre possibilité : antennes parcimonieuses d'accéléromètre en complément d'une mesures sur antenne microphonique : ??\\

\paragraph{\tbullet Revêtement de fibres aramides}


\section{Extraction des composantes tonales et des composantes cyclostationnaires}
cf fiche technique J. Antoni

