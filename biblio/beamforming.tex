%====================================================
\section{Méthodes de formation de voies}
%====================================================

Le principe des méthodes de formation de voies est de pondérer les signaux de mesure à l'aide de vecteurs de pointage de manière à les focaliser dans chaque point du plan sur lequel les sources sont cherchées.
Ces méthodes sont très utilisées car elles offrent beaucoup de flexibilité sur la position des capteurs et sont simples à mettre en œuvre. Cependant, elles offrent une résolution fortement dépendante de la géométrie de l'antenne.\\
\todo[inline]{Manque une référence type review}
Les vecteurs de pointage (correspondant aux lignes de l'opérateur inverse $\bm{W}$) sont les poids attribués à chaque microphone avant de sommer leur réponse.  En tout point focal $i$ du plan de recherche de source, le vecteur de pointage est comparé à la pression mesurée par les microphones. Ainsi, le produit scalaire $\bm{w}_i'\bm{p}$ entre le vecteur de pointage $\bm{w}_i$ conjugué transposé (symbole $'$) et le vecteur des pressions $\bm{p}$ est maximal lorsque les vecteurs sont colinéaires. Le vecteur de pointage est donc associé à un modèle de source. Le modèle de source choisi ici est un ensemble de sources ponctuelles décorrélées. En choisissant un modèle de source décrite par une fonction de Green solution de l'équation d'Helmoltz en champ libre, cette source a pour fonction de transfert du point focal $i$ au microphone $m$ : 
\begin{equation}
	h_{im}=\frac{e^{-jkr_{mi}}}{4\pi r_{mi}}.
\end{equation}
Donc, le vecteur des pression pour une source ponctuelle au point $i$ d'amplitude $A_i$ est $\bm{p}=A_i\bm{h}_i$.\\



%Cette méthode de formation de voies se base sur l'analyse des temps de vol des ondes émises par les sources, dans un milieu dont la propagation est considérée connue. Les retards des signaux sont compensés et sommés pour chaque direction incidente (i.e. point de balayage) possible. La réponse de l'antenne est ainsi maximisée pour l'angle de balayage correspondant à l'angle d'incidence de la source.\\
%L'intensité $I$ de la réponse de l'antenne à un point de balayage $\bm{r}$ est : 
%\begin{equation}
%I(\bm{r})=\sum_{m} \alpha_m(\bm{r}) s_m(t+\tau_m),
%\end{equation}
%où $s_m$ est le signal temporel enregistré par le capteur $m$. $\tau_m$ est le déphasage égal au temps de vol d'une onde se propageant du point d'observation $\bm{r}$ jusqu'au capteur $m$ : $\tau_m=\bm{r}_m / c$, avec $\bm{r}_m$ la distance géométrique du point d'observation $\bm{r}$ à la position du capteur $m$ et $c$ la vitesse de groupe du son dans le milieu d'observation. $\alpha_m$ un terme d'amplitude pouvant contenir une pondération des capteurs ou une correction d'amplitude liée à des pertes, atténuation géométrique, etc. L'intensité est donc maximale quand les signaux retardés sont en phase.\\
%Le terme de déphasage peut également compenser un effet Doppler lorsque la source se déplace à une vitesse connue \citep{Howell1986}, ou encore l'effet d'un écoulement connu sur la propagation de l'onde source\todo{citation : pereira ?}.\\


\todo[inline]{
inconvénient : quantification difficile car chaque source est estimée comme si elle est la seule (decor. 
ref prise en compte des réflexions :
-ajouter la contribution des sources images au processus de formation de voies. B. A. Fenech, “Accurate aeroacoustic measurements in closed-section hard-walled wind tunnels,” Ph.D. dissertation, University of Southampton, June 2009
 
remarque : en beamforming classique, doubler le nombre de micro améliore le RSB de 3db

}

\subsection{Vecteur de pointage indépendant des données \label{bf_standard}}



La formation de voies peut être vue comme la solution d'un problème d'optimisation : afin d'optimiser le vecteur de pointage, on cherche à minimiser l'écart entre l'amplitude estimée $\bm{w}_i'\bm{p}$ et l'amplitude réelle $A_i$. Cette fonction coût est défini à partir d'une densité spectrale $\mathbb{E}\{\bullet\}$ puisque les sources sont des grandeurs aléatoires :
\begin{align}
	J=&\mathbb{E}\left\{ (\bm{w}_i'\bm{p}-A_i)(\bm{w}_i'\bm{p}-A_i)^* \right\}\\
	 =& \bm{w}_i'\bm{S_{pp}}\bm{w}_i-\bm{w}_i'\bm{h}_iG_{ii}-\bm{h}_i'G_{ii}'\bm{w}_i + G_{ii}
\end{align}
$^*$ est l'opérateur conjugué, $\bm{S_{pp}}=\mathbb{E}\{\bm{p}\bm{p}'\}$ et $G_{ii}=E[A_iA_i']$ , soit : 
\begin{equation}
	\frac{\dd J}{\dd \bm{w}_i'}=0 ~~~~~\Leftrightarrow ~~~~~ \bm{w}_i=\frac{\bm{h}_i}{\bm{h}_i'\bm{h}_i}.
\end{equation}

Le vecteur de pointage correspond donc au vecteur des fonctions de transferts normalisé de façon à que l'amplitude $S_i=\bm{w}_i'\bm{p}$ soit égale à 1 quand $\bm{p}=\bm{h}_i$.\\

En présence d'un bruit décorrélé à chaque microphone, on peut montrer que le vecteur de pointage devient : 
\begin{equation}
	\bm{w}_i=\frac{\bm{h}_i}{\bm{h}_i'\bm{h}_i+\gamma},
\end{equation}
avec $\gamma=G_{nn}/G_{ii}$, $G_{nn}$ étant les termes diagonaux de la matrice interspectrale du bruit aux microphones.




\subsection{Construction d'un vecteur de pointage à partir des données}
Certaines méthodes de localisation n'utilisent pas un modèle de source mais construisent le vecteur de pointage à partir de l'ensemble des covariances des signaux de mesure. \\
\cite{Capon1969} propose de minimiser l'énergie en sortie du processeur tout en conservant une contrainte de normalisation que le vecteur de pointage est dans la direction de la source (méthode dite "à variance minimale des sources"): minimiser $\bm{w}_i'\bm{S_{pp}}\bm{w}_i$ (i.e. la densité spectrale des sources) sous la contrainte $\bm{w}_i'\bm{h}_i=1$.
On résout donc, en utilisant le multiplicateur de Lagrange $\lambda$ : 
\begin{equation}
\frac{\dd J}{\dd\bm{w}_i}=0~~~~~\text{et}~~~~~\frac{\dd J}{\dd\lambda}=0
\end{equation}
avec la fonction coût : 
\begin{equation}
J=\bm{w}_i'\bm{S_{pp}}\bm{w}_i + \lambda(\bm{w}_i'\bm{h}_i+\bm{h}_i'\bm{w}_i).
\end{equation}
La résolution de ces 2 équations permet de construire le vecteur de pointage : 
\begin{equation}
	\bm{w}_i=\frac{\bm{S_{pp}}^{-1}\bm{h}_{i}}{\bm{h}_i'\bm{S_{pp}}^{-1}\bm{h}_{i}}.
\end{equation}

Le spectre de puissance de la distribution des sources estimée est alors donné par la relation : 
\begin{align}
	\tilde{G_{ii}} &= \mathbb{E}\{S_iS_i^*\} = \mathbb{E}\{ \bm{w}_i'\bm{pp}'\bm{w}_i\}\\
	&=\frac{1}{\bm{h}_i'\bm{S_{pp}}^{-1}\bm{h}_i}\label{Gii}
\end{align}

L'algorithme MUltiple Signal Classification (MUSIC, \cite{Schmidt1986}) propose une décomposition en valeurs propres de la matrice interspectrale $\bm{S_{pp}}$ pour la décomposer en 2 sous-espaces, l’un associé au signal et l’autre au bruit, afin de diminuer la contribution énergétique du bruit.  Dans l'expression de $\tilde{G_{ii}}$ en équation~\ref{Gii}, $\bm{S_{pp}}$ est remplacé par les composantes correspondant au sous-espace bruit. Ainsi, ce nouvel estimateur sera maximal lorsque le processeur pointe vers une source, puisque les éléments du dénominateur seront décorrélés. Cet estimateur ne correspond alors plus à une densité spectrale des sources mais seulement à un indicateur de présence au point $i$.\\

Ces méthodes font l'hypothèse de sources décorrélées et sont sensibles au non-respect de cette hypothèse. Des stratégies peuvent être mises en place pour prendre en compte la cohérence des sources \citep{Jiang2003}. De plus, l'utilisation des signaux de mesure pour construire le vecteur de pointage rend ce méthode sensibles à la qualité de ce mesures. Pour contourner cette limitation, une pondération peut être ajoutée à la diagonale de la matrice interspectrale \citep{Li2003}.\\

Ces méthodes de formation de voies présentent l'avantages d'être simples à implémenter et relativement rapides à calculer. Mais leur résolution diminue fortement lorsque la longueur d'onde devient grande devant l'écart inter-microphonique et les images présentent alors des lobes secondaires qui rendent les sources difficile à localiser et à séparer. Ce problème peut être résolu par une étape de déconvolution décrite dans la section \ref{deconvolution}.


\todo[inline]{
DORT (pas d'hypothèse sur la distance source-antenne, équation d'euler linéarisées invariantes par RT en changeant le sens de l'écoulement moyen (ex : Localisation de source acoustique en soufflerie anéchoïque par deux techniques d'antennerie : formation de voies et retournement temporel numérique par Thomas Padois))\\

séparation de champ ?
Décomposition en sous-espaces "Orthogonal Beamforming" ?\\
Generalize Inverse Beamforming ?
SAFT, TFM
}
