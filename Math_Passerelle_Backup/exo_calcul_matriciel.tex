\documentclass[12pt,a4paper]{report}
\usepackage[margin=0.5in]{geometry}
\usepackage[french]{babel}
%\usepackage{fontspec}
\usepackage{graphicx}
\usepackage{amsthm,amssymb,amsfonts,mathtools}
\usepackage[utf8]{inputenc}
\usepackage{multicol}
\usepackage{multirow}
\usepackage{amssymb}
\usepackage{array}
\usepackage{graphicx}
%\setlength{\columnseprule}{0.5pt}


\newcounter{exo}[section]
\setcounter{exo}{0}
\newenvironment{exo}{\addtocounter{exo}{1}\vspace{0.4cm}\fbox{\textbf{Exercice \theexo:}}\\[-0.2cm]}{}

\usepackage{icomma} %pour le bon espacement dans le nombres décimaux



%%%%%%%%%%%%%%%%%%%%%%%%%%%%%%%%%%%%%%%%%%
\begin{document}
%%%%%%%%%%head%%%%%%%%%%%%%%%%%%%%%%%
%\twocolumn[ 
%  \begin{@twocolumnfalse}
% \begin{center}
% \fbox{\Large \textsc{ Exercices : Calcul matriciel}}
%\end{center}
%    \hrulefill
%\end{@twocolumnfalse}
 % ]
\begin{center}
	\fbox{\Large \textsc{ Exercices : Calcul matriciel}}
\end{center}
 %%%%%%%%%%%%%%%%%%%%%%%%%%%%%%%%%%%%%%%%%%%%%%%%%%%%
%\begin{multicols}{2}
%\end{multicols}
\indent\exo
\begin{enumerate}
	\item $A=
	\begin{pmatrix}
		1&-1\\1&0
	\end{pmatrix}$ et $B=
	\begin{pmatrix}
		2&1\\2&-1
	\end{pmatrix}$. Calculer $AB$ puis $BA$. \\
	\item $A=
	\begin{pmatrix}
		1&2&3\\6&5&4
	\end{pmatrix}$ et $B=
	\begin{pmatrix}
		1&-2\\-2&4\\1&-2
	\end{pmatrix}$. Calculer $AB$
	\item $A=
	\begin{pmatrix}
		2&1&1\\1&1&1\\3&1&0
	\end{pmatrix}$ et $B=
	\begin{pmatrix}
		1&0&-1&2\\1&2&1&-1\\2&1&1&1
	\end{pmatrix}$. Calculer $AB$ et $BA$
	\item $ A=
	\begin{pmatrix}
		1 &1&1\\1&1&1\\1&1&1
	\end{pmatrix}$. Calculer $A^2$
	\item $A=
	\begin{pmatrix}
		1+i&2\\-3i&4
	\end{pmatrix}$ et $B=
	\begin{pmatrix}
		-i&2i\\5&1
	\end{pmatrix}$ où $i=\sqrt{-1}$. Calculer $AB$
	\item $A=
	\begin{pmatrix}
		2&1\\1&3
	\end{pmatrix}$ et $B=
	\begin{pmatrix}
		4\\1
	\end{pmatrix}$. Calculer $AB$ puis $B^TA$
\end{enumerate}

\exo
\begin{enumerate}
	\item Les matrices suivantes sont-elles inversibles ? Si oui, les inverser.\\
	$A=\begin{pmatrix}
		1&1&1\\
		-1&2&1\\
		0&1&-1
	\end{pmatrix}~~~~
	B=\begin{pmatrix}
		3&2&1\\
		-1&2&-1\\
		0&1&-2
	\end{pmatrix}~~~~
	C=\begin{pmatrix}
		1&2&4\\
		1&-2&-1\\
		2&0&3
	\end{pmatrix}$
	\item $A=\begin{pmatrix}
		1&3&\alpha\\2&-1&1\\-1&1&0
	\end{pmatrix}$. Déterminer les valeurs de $\alpha$ pour lesquels $A$ n'est pas inversible.
	\item $A=\begin{pmatrix}
		\cos(\alpha)&\sin(\alpha)\\-\sin(\alpha)&\cos(\alpha)
	\end{pmatrix}$. Calculer le déterminant de $A$.
\end{enumerate}
	

\exo
\begin{enumerate}
	\item Calculer l'inverse de la matrice $M=
	\begin{pmatrix}
		3&2&-1\\1&-1&1\\2&-4&4
	\end{pmatrix}$.
	\item En déduire les solutions des systèmes 
	$\left\{ \begin{array}{rl}
		3x+2y-z&=5\\
		x-y+z&=1\\
		2x-4y+4z&=-3
	\end{array} \right.$ et 
	$\left\{ \begin{array}{rl}
		3x+2y-z&=-1\\
		x-y+z&=3\\
		2x-4y+4z&=2
	\end{array} \right.$

\end{enumerate}

\exo 

Les familles suivantes forment-elles une base dans $\mathbb{R}^3$ ?\\[0.2cm]
\indent $F_1=\left\{\begin{pmatrix}1\\0\\0\end{pmatrix},\begin{pmatrix}0\\0\\1\end{pmatrix},\begin{pmatrix}0\\1\\0\end{pmatrix}\right\}$~~~~
 $F_2=\left\{	\begin{pmatrix}1\\0\\0\end{pmatrix},\begin{pmatrix}0\\2\\0\end{pmatrix},\begin{pmatrix}0\\0\\0\end{pmatrix}\right\}$~~~~
 $F_3=\left\{	\begin{pmatrix}1\\1\\1\end{pmatrix},\begin{pmatrix}1\\-1\\1\end{pmatrix},\begin{pmatrix}0\\0\\1\end{pmatrix}\right\}$\\[0.2cm]
\indent $F_4=\left\{	\begin{pmatrix}5\\1\\0\end{pmatrix},\begin{pmatrix}0\\2\\8\end{pmatrix},\begin{pmatrix}4\\2\\-1\end{pmatrix},\begin{pmatrix}1\\2\\1\end{pmatrix}\right\}$~~~~
$F_5=\left\{	\begin{pmatrix}1\\1\\1\end{pmatrix},\begin{pmatrix}1\\-1\\1\end{pmatrix},\begin{pmatrix}2\\0\\2\end{pmatrix}\right\}$~~~~


\exo

Soit $\mathcal{T}$ l'application linéaire de matrice $A=\begin{pmatrix} 1&0&0\\0&1&0\\0&0&0\end{pmatrix}$. 
\begin{enumerate}
	\item Quelle est l'image par $\mathcal{T}$ des vecteurs $(1,2,3)^t$ et $(1,2,-1)^t$ ?
	\item Trouver tous les vecteurs $X$ tels que $AX=(1,2,0)^t$.
\end{enumerate}

\exo\\
\indent Soit une application linéaire de matrice $A=\begin{pmatrix}
		0,1&8,9&17,9&17,7\\
		2,3&5,4&13,1&8,5\\
		-1&1,1&1,2&3,2\\
		0,1&2,3&4,7&4,5
	 \end{pmatrix}$.\\

 %c=a+2*b et d=-2*a+c,  avec a, b c et d les colonne de A
\indent Les valeurs propres de cette matrice sont $\lambda_1=12,4855$, $\lambda_2=-1.2855$ et $\lambda_3=\lambda_4=0$. On souhaite résoudre $AX=Y$.
\begin{enumerate}
	\item Connaissant $Y$, que peut-on dire du vecteur inconnu $X$ ?
	\item Que peut-on dire de $A$ exprimée dans la base des vecteurs propres ?
	\item On appelle \textbf{rang} la dimension de l'espace vectoriel engendré par une famille de vecteurs. Le rang est donc le nombre de vecteurs linéairement indépendants d'une famille. Quel est le rang de $A$ (i.e. quel est le rang de la famille de ses vecteurs colonnes) ?
\end{enumerate}

%Si on veut trouver x, on peut pas car A non inversible. Il faut trouver 2 equation supplémentaires.
%Dans la base de VP, on peut réduire la dimension de A.


\exo
\begin{enumerate}
        \item Diagonaliser si cela est possible les matrices suivantes  :\\
        $A=\begin{pmatrix}
		2&1\\3&4
        \end{pmatrix}~~~~
         B=\begin{pmatrix}
		6&-4\\4&-2
	 \end{pmatrix}~~~~
	 C= \begin{pmatrix}
		0&1\\-1&0	
	\end{pmatrix}~~~~
	D=\begin{pmatrix}
		1&1\\-1&-1
	\end{pmatrix}~~~~
	E=\begin{pmatrix}
		1&1&3\\
		1&5&1\\
		3&1&1
	\end{pmatrix}$
	\item Calculer $C^2$, $C^n$.
\end{enumerate}



%%%%%%%%%%%%%%%%%%%%%%%%%%%%%%%%%%%%%%%%%%%%%%%%%%%%%%%%%%%%%%%%%%%%%
\begin{center}
	\fbox{\Large \textsc{ Exercices supplémentaires}}
\end{center}
\indent\exo
\begin{enumerate}
	\item $A=\begin{pmatrix}
		1&2&0\\
		2&1&0\\
		0&1&0
	\end{pmatrix}$. Trouver toutes les matrices $B$ telles que $AB=0_3$.
%	\item Résoudre le système 
%	$\left\{ \begin{array}{rl}
%		2x+y+z&=3\\
%		x-y+3z&=8\\
%		x+2y+2z&=-3
%	\end{array} \right.$.
	\item Soient $a$ et $b$ des réels non nuls. $A=\begin{pmatrix}
		a&b\\0&a
	\end{pmatrix}$. Trouver toutes les matrices $B \in M_2(\mathbb{R})$ qui commutent avec $A$, c'est à dire telles que $AB=BA$.
	%solution : B=(c1 c2 0 c1)'

\end{enumerate}

\exo

Diagonaliser si possible les matrices suivantes : \\
$A=\begin{pmatrix}
	3&6&-4\\
	0&-1&1\\
	3&5&-3
\end{pmatrix}~~~~
B=\begin{pmatrix}
	1&0&0\\
	0&6&-4\\
	0&4&2-
\end{pmatrix}~~~~
C=\begin{pmatrix}
	2&-1&-1\\
	-1&2&-1\\
	0&0&1	
\end{pmatrix}$



%%%%%%%%%%%%%%%%%%%%%%%%%%%%%%%%%%%%%%%%%%%%%%%%%%%%%%%%%%%%%%
\end{document}